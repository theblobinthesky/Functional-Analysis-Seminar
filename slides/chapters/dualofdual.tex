%%%%%%%%%%%%%%%%%%%%%%%%%%%%%%%%%%%%%%%%%%%%%%%%%%%%%%%%%%%%%%%%%%%%%%%%%%%%%%%
\subsection{The dual space of the dual space}
%%%%%%%%%%%%%%%%%%%%%%%%%%%%%%%%%%%%%%%%%%%%%%%%%%%%%%%%%%%%%%%%%%%%%%%%%%%%%%%

\newcommand{\canonicalembinv}[1]{\ensuremath{\inv{(J_#1|_{#1 \to J_#1(#1)})}}}

\begin{frame}{Definition: Bidual Space}
    \begin{definition}
        Let $X$ be a normed \thefield\ vector space.
        \begin{enumeratetheorem}
            \item $X''$ is called the bidual space.
            \item Let $J_X: X \rightarrow X'', p \mapsto (x' \mapsto x'(p))$.
                $J_X$ is called the \highlight{canonical embedding} from $X$ into $X''$.
        \end{enumeratetheorem}
        So $J_X$ returns a function that evaluates dual space elements at $p \in X$.
    \end{definition}
\end{frame}

\begin{frame}{Figure: Dual of Dual}
    \begin{figure}
        \centering
        \includegraphics[width=0.7\linewidth]{../figures/dual_of_dual.pdf}
        \caption{
            An illustration of the main concepts in the dual of duals chapter.
            The operator and bidual operator are in \textcolor{red}{\textbf{red}}.
            The canonical embeddings are in \textcolor{blue}{\textbf{blue}}.
            Isomorphisms are in \textcolor{teal}{\textbf{green}}.
        }
    \end{figure}
\end{frame}

\begin{frame}{Theorem: Bidual Adjoint Embedding}
    \begin{theorem}
        Let $(X, \norm{X}{\cdot})$ and $(Y, \norm{Y}{\cdot})$ be normed \thefield\ vector spaces.
        Let $T \in \contoperators{X}{Y}$ be a bounded linear operator.
        Then we have: $J_Y \chain T = T'' \chain J_X$.
    \end{theorem}
\end{frame}

\begin{frame}{Theorem: Bidual Adjoint Embedding -- Proof Setup}
    \begin{proof}
        Before the proof, we can avoid confusion by typing out what $T'$ and $T''$ evaluate to:
        \begin{itemize}
            \item $T': Y' \rightarrow X', y' \mapsto y' \chain T$ is the adjoint operator.
            \item $T'': X'' \rightarrow Y'', x'' \mapsto x'' \chain T'$ is the biadjoint operator.
        \end{itemize}

        Now first of all, we need to check that the signatures of both sides of the equation match:
        \begin{itemize}
            \item $J_Y: Y \rightarrow Y''$   and $T: X \rightarrow Y$     means $(J_Y \chain T): X \rightarrow Y''$.
            \item $T'': X'' \rightarrow Y''$ and $J_X: X \rightarrow X''$ means $(T'' \chain J_X): X \rightarrow Y''$.
        \end{itemize}
    \end{proof}
\end{frame}

\begin{frame}{Theorem: Bidual Adjoint Embedding -- Proof Calculation}
    \begin{proof}
        Finally, for $p \in X$ and $y' \in Y'$ we have
        \begin{align*}
            \parens{(J_Y \chain T)(p)}(y')
            &= J_Y(Tp)(y') = y'(Tp)                 \explainstep{subst. $p$ in and apply def. of $J_Y$} \\
            &= (y'T)(p)                             \explainstep{use associativity} \\
            &= (T'y')(p)                            \explainstep{apply def. of $T'$} \\
            &= (x' \mapsto x'(p))(T'y')             \explainstep{pull out subst. function} \\
            &= J_X(p)(T'y')                         \explainstep{recognize this is just $J_X$} \\
            &= T''(J_X(p))(y')                      \explainstep{apply def. use $T''$} \\
            &= \parens{(T'' \chain J_X)(p)}(y')   \explainstep{use \chain\ notation}
        \end{align*}
    \end{proof}
\end{frame}

\begin{frame}{Revision: Canonical Embedding Properties}
    \begin{block}{Revision}
        Let $(X, \norm{}{\cdot})$ be a normed \thefield\ vector space. Then we have
        \begin{enumeratetheorem}
            \item The canonical embedding $J_X$ is an isometric injective function.
            \item $J: X \rightarrow J_X(X), x \mapsto J_X(x)$ is an isometric isomorphism.
            \item $J_X$ is a bounded, linear operator.
            \item \canonicalembinv{X} is a bounded, linear operator.
        \end{enumeratetheorem}
    \end{block}
\end{frame}

\begin{frame}{Revision: Canonical Embedding -- Proof Idea}
    \begin{block}{Proof Idea}
        \begin{enumerateproof}
            \item \refertolecture
            \item Follows by definition of $J$ and from ''i)''.
            \item
                Since $J: X \rightarrow J_X(X), x \mapsto J_X(x)$ is an isometric isomorphism, we have
                $J_X$ is linear since the $x' \in X'$ are linear, and
                $\norm{}{J_X} = 1$ means $J_X$ is bounded.
            \item
                Since $J: X \rightarrow J_X(X), x \mapsto J_X(x)$ is an isometric isomorphism, we have
                \canonicalembinv{X} inherits linearity from $J$, and
                $\norm{}{J_X} = 1$ therefore $\norm{}{\canonicalembinv{X}} = 1$ means \canonicalembinv{X} is bounded.
        \end{enumerateproof}
    \end{block}
\end{frame}

\begin{frame}{Corollary: Strengthened Bidual Adjoint Embedding}
    \begin{corollary}
        We can strengthen the theorem using the isomorphism: \\
        Let $(X, \norm{X}{\cdot})$ and $(Y, \norm{Y}{\cdot})$ be normed \thefield\ vector spaces.
        Let $T \in \contoperators{X}{Y}$ be a bounded linear operator.
        Then we have
        \begin{enumeratetheorem}
            \item $J_Y \chain T = T'' \chain J_X$.
            \item $T''|_{X \to J_X(X)} = J_Y \chain T \chain \canonicalembinv{X}$
            \item $T = \canonicalembinv{Y} \chain T'' \chain J_X$.
        \end{enumeratetheorem}
    \end{corollary}
    \begin{proof}
        \begin{enumerateproof}
            \item This is the statement of the theorem.
            \item The isomorphism from the revision gives us the result.
            \item The isomorphism from the revision gives us the result.
        \end{enumerateproof}
    \end{proof}
\end{frame}

\begin{frame}{Theorem: Characterization of Adjoint Operators}
    \begin{theorem}
        Let $S \in \contoperators{Y'}{X'}$ be a continuous, linear operator. Then we have
        \[ \exists T \in \contoperators{X}{Y}: T' = S \text{ if and only if } S'(J_X(X)) \subset J_Y(Y) \]
    \end{theorem}
\end{frame}

\begin{frame}{Theorem: Characterization of Adjoint Operators -- Proof ($\Rightarrow$)}
    \begin{proof}
        \leftrightproof We have
        \begin{align*}
            S'(J_X(X))
            &= T''(J_X(X))      \explainstep{substitute in $S = T'$} \\
            &= J_Y(T(X))        \explainstep{use the theorem} \\
            &\subset J_Y(Y)     \explainstep{use $T(X) \subset Y$}
        \end{align*}
    \end{proof}
\end{frame}

\begin{frame}{Theorem: Characterization of Adjoint Operators -- Proof ($\Leftarrow$) Setup}
    \begin{proof}
        \rightleftproof
        We have $S'(J_X(X)) \subset J_Y(Y)$ and the revision gives us $J_Y(Y) \isomorphto Y$. \\
        So for $x \in X$ and $y''_x = S'(J_X(x))$ there is a (unique) $y_x \in Y$ with $y''_x = J_Y(y_x)$. \\
        Choose $ T: X \rightarrow Y, x \mapsto y_x $
        We know $T$ exists (and is unique) due to the previous argument.

        We know $T$ is linear and continuous, as
        $ T = y_\cdot = \canonicalembinv{Y} \chain S' \chain J_X $
        and all elements in the chain are bounded, linear operators.
    \end{proof}
\end{frame}

\begin{frame}{Theorem: Characterization of Adjoint Operators -- Proof ($\Leftarrow$) $S = T'$}
    \begin{proof}
        Let $y' \in Y'$ and $x \in X$. Lastly, we need to prove that $S = T'$:
        \begin{align*}
            (Sy')(x)
            &= J_X(x)(S y')                                             \explainstep{express using $J_X$} \\
            &= (S' J_X(x))(y') = (S' \chain J_X)(x)(y')                 \explainstep{apply def. of ajoint op.} \\
            &= J_Y((\canonicalembinv{Y} \chain S' \chain J_X)(x))(y')   \explainstep{use $J_Y \chain \canonicalembinv{Y} = \id$} \\
            &= y'((\canonicalembinv{Y} \chain S' \chain J_X)(x))        \explainstep{evaluate $J_Y$ at the vector} \\
            &= y'(Tx) = (y' \chain T)(x)                                \explainstep{subst. in $T = J_Y^{-1} \chain S' \chain J_X$} \\
            &= (T' y')(x)                                               \explainstep{apply def. of adjoint op.}
        \end{align*}
    \end{proof}
\end{frame}
