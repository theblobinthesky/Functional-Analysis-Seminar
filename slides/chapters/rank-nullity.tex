%%%%%%%%%%%%%%%%%%%%%%%%%%%%%%%%%%%%%%%%%%%%%%%%%%%%%%%%%%%%%%%%%%%%%%%%%%%%%%%
\subsection{The rank-nullity theorem for operators}
%%%%%%%%%%%%%%%%%%%%%%%%%%%%%%%%%%%%%%%%%%%%%%%%%%%%%%%%%%%%%%%%%%%%%%%%%%%%%%%

\newcommand{\anihv}{\ensuremath{V_\perp}}

\begin{frame}{Definition: Annihilator}
    \begin{definition}
        Let $(X, \norm{}{\cdot})$ be a normed \thefield\ vector space and $V \subset X'$.
        Then we define the annihilator of $V$ in $X$ as
        \[ \anihv = \braces{x \in X \mid \forall x' \in V: x'(x) = 0} \]
        \sidenote{
            The annihilator is the set of linear, bounded functionals
            that "see" exactly the opposite of $V$
            and are "blind" to $V$.
        }
    \end{definition}
\end{frame}

\begin{frame}{Revision: Hahn-Banach Corollary}
    \begin{block}{Revision}
        Let $(X, \norm{}{\cdot})$ be a normed \thefield\ vector space,
        $U \subset X$ a closed subspace and $x \in X \setminus U$.
        Then we have
        \[ \exists x' \in X': x'|_U = 0 \logicaland x'(x) \neq 0 \]
    \end{block}
\end{frame}

\begin{frame}{Theorem: Annihilator is Closed Linear Subspace}
    \begin{theorem}
        Let $(X, \norm{}{\cdot})$ be a normed \thefield\ vector space and $V \subset X'$.
        Then we have
        \[ \anihv \subset X \text{ is a closed linear subspace} \]
    \end{theorem}
    \begin{proof}
        We have
        $ \anihv = \bigcap_{x' \in V}{\inv{(x')}(0)} $
        As an intersection of closed sets, \anihv must be closed.
    \end{proof}
\end{frame}

\newcommand{\kerdualperp}{\ensuremath{(\ker T')_\perp}\xspace}

\begin{frame}{Theorem: Rank-Nullity Generalized}
    \begin{theorem}
        Let $T \in \contoperators{X}{Y}$ be a bounded, linear operator.
        Then we have \[ \overline{\im T} = \kerdualperp \]
        \sidenote{
            In linear algebra lectures this is proven for finite-dimensional vector spaces
            (see Satz 6.1.5\ \cite{werner_funktionalanalysis_2018})
        }
    \end{theorem}
\end{frame}

\begin{frame}{Theorem: Rank-Nullity Generalized -- Proof ($\subset$)}
    \begin{block}{Proof}
        \leftrightinclusion
        Let $Tx \in \im T$ with $x \in X$ and $y' \in \ker T'$. \\
        We first prove $Tx \in \kerdualperp$:
        \begin{align*}
            y'(Tx) &= (y' \chain T)(x)  \explainstep{associativity and use \chain\ notation} \\
            &= (T' y')(x)               \explainstep{apply adjoint op. definition} \\
            &= 0(x) = 0                 \explainstep{apply $y' \in \ker T'$}
        \end{align*}
        Since this holds for all choices of $Tx$ we get $ \im T \subset \kerdualperp $
        Since \kerdualperp is closed, we also get $ \overline{\im T} \subset \kerdualperp $
    \end{block}
\end{frame}

\begin{frame}{Theorem: Rank-Nullity Generalized -- Proof ($\supset$)}
    \begin{proof}
        \rightleftinclusion
        We can prove the contraposition
        $ \parens{Y \setminus \overline{\im T}} \subset \parens{Y \setminus \kerdualperp} $
        Set $U = \overline{\im T}$ and let $y \in Y \setminus U$.
        We know $U$ that a closed linear subspace.
        The corollary of the theorem of Hahn-Banach tells us that
        $ \exists y' \in Y' : y'|_U = 0 \logicaland y'(y) \neq 0 $
        Since $\ker T' \subset Y'$
        and $\forall y' \in \ker T' : y'(y) \neq 0$
        we get $ y \in Y \setminus \kerdualperp $
    \end{proof}
\end{frame}

\begin{frame}{Corollary: Operator Solutions}
    \begin{corollary}
        Let $T \in \contoperators{X}{Y}$ be a linear, continuous operator with $\im T$ closed. Then we have
        \[ y \in \im T \text{ if and only if } \forall y' \in Y': T'y' = 0 \implies y'(y) = 0 \]
    \end{corollary}
    \begin{proof}
        We have
        \begin{align*}
            y \in \im T
            &= \overline{\im T}                                     \explainstep{$\im T$ is closed} \\
            &= \kerdualperp                                         \explainstep{apply the theorem} \\
            \iff &\forall y' \in \ker T' : y'(y) = 0                \explainstep{apply def. of annihilator} \\
            \iff &\forall y' \in Y': T'y' = 0 \implies y'(y) = 0    \explainstep{write in equivalent way}
        \end{align*}
    \end{proof}
\end{frame}

\begin{frame}{Example: Left Shift}
    \begin{example}
        Consider the left shift operator $T: \lp \to \lp$ from earlier.
        Its adjoint $T': \lpconj \to \lpconj$ is the right shift operator:
        $ T'(s_1, s_2, \dots) = (0, s_1, s_2, \dots) $

        To apply the theorem, we first find $\ker T'$.
        If $T's = 0$, then $(0, s_1, s_2, \dots) = (0, 0, 0, \dots)$, which implies $s = 0$.
        Thus, $\ker T' = \{0\}$.

        The theorem states that $\overline{\im T} = (\ker T')_\perp = \{0\}_\perp = \lp$.
        This is consistent with the fact that the left shift is surjective ($\im T = \lp$).
    \end{example}
\end{frame}

\begin{frame}{Example: Right Shift}
    \begin{example}
        Consider the right shift operator $T: \lp \to \lp, T(x_1, x_2, \dots) = (0, x_1, x_2, \dots)$.
        Its image is the closed subspace $ \im T = \{ y \in \lp \mid y_1 = 0 \} $.

        The adjoint $T': \lpconj \to \lpconj$ is the left shift, $T'(s_1, s_2, s_3, \dots) = (s_2, s_3, \dots)$.
        Thus $\ker T' = \{ (s_1, 0, 0, \dots) \mid s_1 \in \thefield \} = \text{span}\{e_1'\}$.

        By the corollary, $y \in \im T$ iff it is annihilated by every functional in $\ker T'$.
        For any $s \in \ker T'$, the condition $s(y) = 0$ becomes:
        $ \sum_{k=1}^\infty y_k s_k = y_1 s_1 = 0 $

        Since this must hold for all $s_1 \in \thefield$, we must have $y_1 = 0$.
        This matches $\im T$, verifying the theorem.
    \end{example}
\end{frame}
