%%%%%%%%%%%%%%%%%%%%%%%%%%%%%%%%%%%%%%%%%%%%%%%%%%%%%%%%%%%%%%%%%%%%%%%%%%%%%%%
\subsection{The rank-nullity theorem for operators}
%%%%%%%%%%%%%%%%%%%%%%%%%%%%%%%%%%%%%%%%%%%%%%%%%%%%%%%%%%%%%%%%%%%%%%%%%%%%%%%

\newcommand{\anihv}{\ensuremath{V_\perp}}

\begin{frame}{Definition: Annihilator}
    \begin{definition}
        Let $(X, \norm{}{\cdot})$ be a normed \thefield\ vector space, $U \subset X$ and $V \subset X'$.
        Then we define the annihilator of $V$ in $X$ as
        \[ \anihv = \braces{x \in X \mid \forall x' \in V: x'(x) = 0} \]
    \end{definition}
\end{frame}

\begin{frame}{Context}
    To prove properties about this set, we need another corollary of the theorem of Hahn-Banach.
\end{frame}

\begin{frame}{Revision: Hahn-Banach Corollary}
    \begin{block}{Revision}
        Let $(X, \norm{}{\cdot})$ be a normed \thefield\ vector space,
        $U \subset X$ a closed subspace and $x \in X \setminus U$.
        Then we have
        \[ \exists x' \in X': x'|_U = 0 \logicaland x'(x) \neq 0 \]
    \end{block}
\end{frame}

\begin{frame}{Revision: Hahn-Banach Corollary -- Proof}
    \begin{proof}
        Let $Y = X / U$ be the (canonical) quotient space.
        Then $Y$ is a normed \thefield\ vector space. (todo: why?)
        Set $y = x \in Y$.
        We can apply the theorem of Hahn-Banach
        to obtain $y' \in Y'$ with $y'(y) \neq 0$ and $y'|_U = 0$.
    \end{proof}
\end{frame}

\begin{frame}{Theorem: Annihilator is Closed Linear Subspace}
    \begin{theorem}
        Let $(X, \norm{}{\cdot})$ be a normed \thefield\ vector space, $U \subset X$ and $V \subset X'$.
        Then we have
        \[ \anihv \subset X \text{ is a closed linear subspace} \]
    \end{theorem}
\end{frame}

\begin{frame}{Theorem: Annihilator is Closed Linear Subspace -- Proof}
    \begin{proof}
        We have
        \[ \anihv = \bigcap_{x' \in V}{\inv{(x')}(0)} \]
        As an intersection of closed sets, \anihv must be closed.
    \end{proof}
\end{frame}

\newcommand{\kerdualperp}{\ensuremath{(\ker T')_\perp}\xspace}

\begin{frame}{Theorem: Rank-Nullity Generalized}
    \begin{theorem}
        Let $T \in \contoperators{X}{Y}$ be a bounded, linear operator.
        Then we have \[ \overline{\im T} = \kerdualperp \]
        \sidenote{
            In linear algebra lectures this is proven for finite-dimensional vector spaces
            (see Satz 6.1.5\ \cite{werner_funktionalanalysis_2018})
        }
    \end{theorem}
\end{frame}

\begin{frame}{Theorem: Rank-Nullity Generalized -- Proof ($\subset$)}
    \begin{proof}
        \leftrightinclusion
        Let $Tx \in \im T$ with $x \in X$ and $y' \in \ker T'$. \\
        We first prove $Tx \in \kerdualperp$:
        \begin{align*}
            y'(Tx) &= (y' \chain T)(x)  \explainstep{associativity and use \chain\ notation} \\
            &= (T' y')(x)               \explainstep{apply adjoint op. definition} \\
            &= 0(x) = 0                 \explainstep{apply $y' \in \ker T'$}
        \end{align*}
        Since this holds for all choices of $Tx$ we get \[ \im T \subset \kerdualperp \]
        Since \kerdualperp is closed, we also get \[ \overline{\im T} \subset \kerdualperp \]
    \end{proof}
\end{frame}

\begin{frame}{Theorem: Rank-Nullity Generalized -- Proof ($\supset$)}
    \begin{proof}
        \rightleftinclusion
        We can prove the contraposition
        \[ \parens{Y \setminus \overline{\im T}} \subset \parens{Y \setminus \kerdualperp} \]
        Set $U = \overline{\im T}$ and let $y \in Y \setminus U$.
        We know $U$ that a closed linear subspace.
        The corollary of the theorem of Hahn-Banach tells us that
        \[ \exists y' \in Y' : y'|_U = 0 \logicaland y'(y) \neq 0 \]
        Since $\ker T' \subset Y'$
        and \[\forall y' \in \ker T' : y'(y) \neq 0 \]
        we get \[ y \in Y \setminus \kerdualperp \]
    \end{proof}
\end{frame}

\begin{frame}{Corollary: Operator Solutions}
    \begin{corollary}
        Let $T \in \contoperators{X}{Y}$ be a linear, continuous operator with $\im T$ closed. Then we have
        \[ y \in \im T \text{ if and only if } \forall y' \in Y': T'y' = 0 \implies y'(y) = 0 \]
    \end{corollary}
\end{frame}

\begin{frame}{Corollary: Operator Solutions -- Proof}
    \begin{proof}
        We have
        \begin{align*}
            y \in \im T
            &= \overline{\im T}                                     \explainstep{$\im T$ is closed} \\
            &= \kerdualperp                                         \explainstep{apply the theorem} \\
            \iff &\forall y' \in \ker T' : y'(y) = 0                \explainstep{apply def. of annihilator} \\
            \iff &\forall y' \in Y': T'y' = 0 \implies y'(y) = 0    \explainstep{write in equivalent way}
        \end{align*}
    \end{proof}
\end{frame}

\begin{frame}{Example: Shift Operator Revisited}
    \begin{example}
        We can refer back to the example with the left shift operator. \\
        The left shift operator $T$ has $\im T = \lp$, which is (trivially) closed. \\
        Since we already know the adjoint operator
        we can verify that the last theorem works. \\
        Everything is up to isometry:
        Let $\sequence{k}{x_k} \in \lp$ such that $T\parens{\sequence{k}{x_k}} = (x_2, \dots) \in \lp$.
        Then \[ T'y' = 0 \implies y'(y) = 0 \]
        amounts to "if the right shifted version of a sequence is 0 then
    \end{example}
\end{frame}
