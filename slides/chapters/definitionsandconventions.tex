%%%%%%%%%%%%%%%%%%%%%%%%%%%%%%%%%%%%%%%%%%%%%%%%%%%%%%%%%%%%%%%%%%%%%%%%%%%%%%%
\subsection{The basic definitions and conventions}
%%%%%%%%%%%%%%%%%%%%%%%%%%%%%%%%%%%%%%%%%%%%%%%%%%%%%%%%%%%%%%%%%%%%%%%%%%%%%%%

\begin{frame}{Terminology}
    The terms \highlight{adjoint} and \highlight{dual} are often used interchangeably.
    We will standardize to \highlight{adjoint}, to avoid unnecessary confusion.
    Following\ \cite{werner_funktionalanalysis_2018}, we write $\thefield \in \{\reals, \complexes\}$ when the field is unspecified.
\end{frame}

\begin{frame}{Definition: Linear Operator}
    \begin{definition}
        We remind ourselves of the following concepts from the lecture \cite{krieg_functional_2025}:
        Let \normedspace{X}, \normedspace{Y} and \normedspace{Z} be normed \thefield-vector spaces
        \begin{enumeratetheorem}
            \item
                Let \[ T: (X, \norm{X}{\cdot}) \rightarrow (Y, \norm{Y}{\cdot}) \] be a linear mapping.
                Then we call $T$ a \highlight{linear operator}.
                We call $T$ \highlight{bounded}, if \[\exists C > 0, \forall x \in X: \norm{Y}{Tx} \le C \norm{X}{x}\]
                For brevity, we will use the notation $T: X \rightarrow Y$ for linear operators rather than \\
                $T: (X, \norm{X}{\cdot}) \rightarrow (Y, \norm{Y}{\cdot})$.
            \item The (topological) dual space is defined as \[ X' \defeq \dualspace{X} \]
            \item The closed unit ball in \normedspace{X} is abbreviated with \closedunitball{X}.
        \end{enumeratetheorem}
    \end{definition}
\end{frame}

\begin{frame}{Revision: Foundational Statements}
    \begin{block}{Revision}
        The following statements are foundational for this topic: \\
        Let \twonormedspaces.
        \begin{enumeratetheorem}
            \item The set of continuous, linear operators \contoperators{X}{Y} is a Banach space if and only if $Y$ is a Banach space.
                In particular, the topological dual space \dualspace{X} is a Banach space.
            \item Let $T: X \to Y$ be a linear operator.
                Then $T$ is bounded if and only if $T$ is continous.
            \item Let \normedspace{Z} be a normed \thefield-vector space,
                let $T: X \to Y$ be a linear, bounded operator and
                let $S: Y \to Z$ be a linear, bounded operator.
                Then $S \chain T$ is a linear, bounded operator.
            \item Let \normedspace{Z} be a normed \thefield-vector space,
                let $T: X \to Y$ be a linear operator and
                let $S: Y \to Z$ be a linear operator.
                If $T$ or $S$ is compact, $S \chain T$ is compact.
        \end{enumeratetheorem}
    \end{block}
\end{frame}

\begin{frame}{Definition: Adjoint Operator}
    \begin{definition}
        Let \twonormedspaces\ and $T \in \contoperators{X}{Y}$. \\
        Then $T': Y' \rightarrow X', y' \mapsto y' \chain T$ is called the adjoint operator.
        From now on, we will implicitly refer to the normed \thefield\ vector spaces $X, Y$ and the topological dual spaces $X', Y'$ when talking about the dual operator $T'$.
    \end{definition}
    \begin{block}{Remark}
        In comparison to the operators we have previously worked with, the adjoint operator takes and outputs linear, bounded operators.
        It's one more level of abstraction removed from \thefield. \\
        For $y' \in Y' = \dualspace{Y}$, the adjoint operator evaluates to $T' y' \in \dualspace{X} = X'$.
    \end{block}
\end{frame}

\begin{frame}{Example: Dual Space of $\reals^n$}
    \begin{example}
        Let $n \in \naturals$. Then we have
        \begin{align*}
            (\reals^n)'
            &= \braces{g: \reals^n \to \reals \text{ linear, continuous}}   \explainstep{apply def. of dual space} \\
            &= \braces{g: \reals^n \to \reals \text{ linear}}               \explainstep{result from linear algebra} \\
            &\isomorphto \reals^n                                           \explainstep{result from linear algebra see \cite{fischer_bilinearformen_2025}}
        \end{align*}
    \end{example}
\end{frame}

\begin{frame}{Example: Adjoint of Matrix Operator}
    \begin{example}
        Let $m \in \naturals$ and let $A \in \reals^{m \times n}$. \\
        Consider the function $f: \reals^n \to \reals^m, x \mapsto Ax$.
        The signature of the adjoint is $f': (\reals^m)' \to (\reals^n)'$.
        With $e_i$ we denote the standard basis vectors
        and with $e_i'$ we denote the dual standard basis vectors.

        Let $i = 1, \dots, m$ and $j = 1, \dots, n$. We have
        \begin{align*}
            (f' e_i')(e_j)
            &= (e_i' \chain f)(e_j)         \explainstep{apply def. of adjoint op.} \\
            &= f(e_j)_i = a_{ij}            \explainstep{apply def. of $e_i'$ and $f$}
        \end{align*}

        So if we set $i = 1$ for instance,
        we get $(f' e_1')(e_j) = a_{1j}$.
        The first "column" of $f'$ (up to isomorphism)
        must be $(a_{1j})_{j = 1,\dots,n}$.
        In conclusion, we have (up to isomorphism):
        \[ f': \reals^m \to \reals^n, x \mapsto A^T x \]
    \end{example}
\end{frame}
