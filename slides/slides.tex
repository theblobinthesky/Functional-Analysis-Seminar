\documentclass[aspectratio=169]{beamer}

% Theme configuration - cleaner academic look
\usetheme{Madrid}
\usecolortheme{whale}
\useinnertheme{rectangles}
\useoutertheme{infolines}

% Remove navigation symbols
\setbeamertemplate{navigation symbols}{}
\setbeamertemplate{caption}[numbered]

% Custom colors based on University of Passau branding
\definecolor{unipassaublue}{RGB}{0,62,107}
\definecolor{unipassaugray}{RGB}{88,88,90}
\setbeamercolor{structure}{fg=unipassaublue}
\setbeamercolor{frametitle}{bg=unipassaublue,fg=white}
\setbeamercolor{title}{fg=white,bg=unipassaublue}
\setbeamercolor{block title}{bg=unipassaublue,fg=white}
\setbeamercolor{block body}{bg=unipassaublue!10}
\setbeamercolor{item}{fg=unipassaublue}

% Logo in footer
\logo{\includegraphics[height=0.8cm]{../uni-passau-logo.png}}

% Packages
\usepackage[english]{babel}
\usepackage{amsmath, amsfonts, amsthm, mathtools}
\usepackage{dsfont}
\usepackage{graphicx}
\usepackage{booktabs}
\usepackage{hyperref}
\usepackage{caption}
\usepackage{xspace}
\usepackage{enumitem}
\usepackage{xcolor}
\usepackage[
    backend=bibtex,
    style=alphabetic,
]{biblatex}
\addbibresource{../bibliography.bib}

% Macros from writeup/writeup.tex
\newcommand{\defeq}{\vcentcolon=}
\newcommand{\eqdef}{=\vcentcolon}
\newcommand{\isomorphto}{\cong}
\renewcommand{\implies}{\Rightarrow}
\newcommand{\convergesto}{\longrightarrow}

\newcommand{\normedspace}[1]{\ensuremath{(#1, \|\cdot\|_{#1})}}
\newcommand{\twonormedspaces}{\normedspace{X} and \normedspace{Y} be normed \thefield-vector spaces}

\newcommand{\sequence}[2]{\ensuremath{(#2)_{#1 \in \naturals}}}
\newcommand{\highlight}[1]{\textcolor{blue}{#1}}
\newcommand{\norm}[2]{\|#2\|_{#1}}
\newcommand{\abs}[1]{|#1|}
\newcommand{\contoperators}[2]{\ensuremath{\mathcal{L}\left(#1, #2\right)}}
\newcommand{\dualspace}[1]{\contoperators{#1}{\thefield}}
\newcommand{\closedunitball}[1]{\ensuremath{\overline{B}_#1}}
\newcommand{\thefield}{\ensuremath{\mathbb{F}}}
\newcommand{\chain}{\ensuremath{\circ}}
\newcommand{\explainstep}[1]{&& \textcolor{darkgray}{\text{(#1)}}}
\newcommand{\logicaland}{\ensuremath{\wedge\;}}
\newcommand{\logicalor}{\ensuremath{\lor}}
\newcommand{\inv}[1]{\ensuremath{#1^{-1}}}

% ===== COMMUNICATION ======
\newcommand{\sidenote}[1]{\textup{\textcolor{blue}{$\triangleright$ #1}}}

% ===== PROOF DIRECTIONS =====
\newcommand{\leftrightproof}{"\ensuremath{\Rightarrow}":\space}
\newcommand{\rightleftproof}{"\ensuremath{\Leftarrow}":\space}
\newcommand{\leftrightinclusion}{"\ensuremath{\subset}":\space}
\newcommand{\rightleftinclusion}{"\ensuremath{\supset}":\space}

% ===== BRACKETS =====
\newcommand{\parens}[1]{\left(#1\right)}
\newcommand{\braces}[1]{\left\{#1\right\}}

% ===== SETS ======
\newcommand{\reals}{\ensuremath{\mathbb{R}}\xspace}
\newcommand{\complexes}{\ensuremath{\mathbb{C}}\xspace}
\newcommand{\naturals}{\ensuremath{\mathbb{N}}\xspace}

% ===== SPECIAL OPERATORS =====
\DeclareMathOperator{\rk}{rk}
\let\ker\relax
\DeclareMathOperator{\ker}{Ker}
\DeclareMathOperator{\im}{Im}
\DeclareMathOperator{\id}{Id}

% ===== ENUMERATIONS =====
\newlist{enumeratetheorem}{enumerate}{1}
\setlist[enumeratetheorem]{label=\roman*), leftmargin=*, labelsep=0.2em, align=left}

\newlist{enumerateproof}{enumerate}{1}
\setlist[enumerateproof]{label="\roman*)":\space, leftmargin=0em, labelwidth=*, align=left}

% ===== MISC =====
\newcommand{\refertolecture}{\text{Please refer to the functional analysis lecture notes\ \cite{krieg_functional_2025} from SS/2025.}}

% Theorem environments for beamer
\setbeamertemplate{theorems}[numbered]

\title{Dual Operators}
\subtitle{Seminar Functional Analysis}
\author{Erik Stern}
\institute{University of Passau}
\date{January 2026}

% No section/subsection title slides - section shown in footer via infolines theme

\begin{document}

% Title frame
{
\logo{\includegraphics[height=0.8cm]{../uni-passau-logo.png}}
\begin{frame}
    \titlepage
\end{frame}
}


% Table of contents
\begin{frame}{Outline}
    \tableofcontents
\end{frame}

%%%%%%%%%%%%%%%%%%%%%%%%%%%%%%%%%%%%%%%%%%%%%%%%%%%%%%%%%%%%%%%%%%%%%%%%%%%%%%%
\section{Motivation}
%%%%%%%%%%%%%%%%%%%%%%%%%%%%%%%%%%%%%%%%%%%%%%%%%%%%%%%%%%%%%%%%%%%%%%%%%%%%%%%
\begin{frame}{Motivation}
    The goal of dual- or adjoint operators is to generalize the notion of an adjoint matrix
    (often denoted as $A^T$ over \reals\ or $A^H$ over \complexes)
    to operators on normed \reals or \complexes vector spaces.

    We will cover the following topics:
    \begin{itemize}
        \item The operator $\cdot^H$ is linear and isometric wrt. the spectral norm.
        \item The fundamental theorem of linear algebra (Gilbert Strang) or rank-nullity theorem: \\
            $(\im A)^\perp = \ker A^H$.
        \item TODO Add more
    \end{itemize}
\end{frame}


%%%%%%%%%%%%%%%%%%%%%%%%%%%%%%%%%%%%%%%%%%%%%%%%%%%%%%%%%%%%%%%%%%%%%%%%%%%%%%%
\section{The adjoint operator}
%%%%%%%%%%%%%%%%%%%%%%%%%%%%%%%%%%%%%%%%%%%%%%%%%%%%%%%%%%%%%%%%%%%%%%%%%%%%%%%

\subsection{The basic definitions and conventions}

\noindent The terms \highlight{adjoint} and \highlight{dual} are often used interchangeably.
We will standardize on \highlight{adjoint}, to avoid unnecessary confusion.
Following\ \cite{werner_funktionalanalysis_2018}, we write $\thefield \in \{\reals, \complexes\}$ when the field is unspecified.

\begin{definition}
    We remind outselves of the following concepts from the lecture \cite{krieg_functional_2025}:
    Let \normedspace{X}, \normedspace{Y} and \normedspace{Z} be normed \thefield-vector spaces
    \begin{romanenum}
        \item
            Let \[ T: (X, \norm{X}{\cdot}) \rightarrow (Y, \norm{Y}{\cdot}) \] be a linear mapping.
            Then we call $T$ a \highlight{linear operator}.
            We call $T$ \highlight{bounded}, if \[\exists C > 0, \forall x \in X: \norm{Y}{Tx} \le C \norm{X}{x}\]
            \note{
                For brevity, we will use the notation $T: X \rightarrow Y$ for linear operators rather than \\
                $T: (X, \norm{X}{\cdot}) \rightarrow (Y, \norm{Y}{\cdot})$.
            }

        \item The (topological) dual space is defined as \[ X' \defeq \dualspace{X} \]
        \item The closed unit ball in \normedspace{X} is abbreviated with \closedunitball{X}.

    \end{romanenum}
\end{definition}

\begin{revision} \label{revision:general-revision}
    The following statements are foundational for this topic: \\
    Let \twonormedspaces.
    \begin{romanenum}
        \item The set of continuous, linear operators \contoperators{X}{Y} is a Banach space if and only if $Y$ is a Banach space.
            In particular, the topological dual space \dualspace{X} is a Banach space.
        \item Let $T: X \to Y$ be a linear operator.
            Then $T$ is bounded if and only if $T$ is continous.
        \item Let \normedspace{Z} be a normed \thefield-vector space,
            let $T: X \to Y$ be a linear, bounded operator and 
            let $S: Y \to Z$ be a linear, bounded operator.
            Then $S \chain T$ is a linear, bounded operator.
        \item Let \normedspace{Z} be a normed \thefield-vector space,
            let $T: X \to Y$ be a linear operator and 
            let $S: Y \to Z$ be a linear operator.
            If $T$ or $S$ is compact, $S \chain T$ is compact.
    \end{romanenum}
\end{revision}
\begin{proof}
    \refertolecture
\end{proof}

\begin{definition}
    Let \twonormedspaces\ and $T \in \contoperators{X}{Y}$. \\
	Then $T': Y' \rightarrow X', y' \mapsto y' \chain T$ is called the adjoint operator.
    From now on, we will implicitly refer to the normed \thefield\ vector spaces $X, Y$ and the topological dual spaces $X', Y'$ when talking about the dual operator $T'$.
\end{definition}

\begin{remark}
    In comparison to the operators we have previously worked with, the adjoint operator takes and outputs linear, bounded operators.
	It's one more level of abstraction removed from \thefield. \\
	For $y' \in Y' = \dualspace{Y}$, the adjoint operator evaluates to $T' y' \in \dualspace{X} = X'$.
\end{remark}

\begin{example}
    % TODO: Add simple integral operator example.
\end{example}
\subsection{The basic properties}

\noindent The proofs will use the Hahn-Banach corollaries a couple of times.
So first, we need to recall Hahn-Banach related theorems:

\begin{revision}[Hahn-Banach]\label{theorem:hahn-banach-op-norm}
    Let $(X, \norm{}{\cdot})$ be a normed space and $0 \neq x \in X$. \\
    Then we have \[\exists f \in X': \norm{}{f} = 1 \logicaland f(x) = \norm{}{x}\]
\end{revision}
\begin{proof}
    \refertolecture
\end{proof}

\begin{revision}
    TODO (Explain): \\
    Let $(X, \norm{}{\cdot})$ be a normed \thefield\ vector space and $x \in X$. \\
    Then we have \[\norm{X}{x} = \sup_{f \in X', \norm{}{f} \le 1}\abs{f(x)}\]
\end{revision}
\begin{proof}
    We will distinguist two cases:
    \begin{itemize}
        \item Case 1 ($x = 0$):
            Since $X'$ contains linear operators,
            \[\norm{X}{x} = 0 = \sup_{f \in X', \norm{}{f} \le 1}\abs{f(0)}\]

        \item Case 2 $x \neq 0$:
            We have
            \[
                \sup_{f \in X', \norm{}{f} \le 1}\abs{f(x)}
                \le \sup_{f \in X', \norm{}{f} \le 1}{\norm{}{f}\norm{X}{x}}
                \le 1 \norm{X}{x} = \norm{X}{x}
            \]
            Using Hahn-Banach from revision \ref{theorem:hahn-banach-op-norm} we get
            \[\exists f \in X': \norm{}{f} = 1 \logicaland \abs{f(x)} = \norm{X}{x}\]
            So we get \[\sup_{f \in X', \norm{}{f} \le 1}\abs{f(x)} = \norm{X}{x}\]

    \end{itemize}
\end{proof}

\begin{theorem}
    The adjoint operator has the following properties:
    \begin{enumeratetheorem}
        \item $T' \in \contoperators{Y'}{X'}$, so $T'$ is linear and bounded. \\
            \note{This implies $\forall y' \in \contoperators{Y'}{X'}: T'y' \in X'$.}
        \item $T \mapsto T'$ is linear and isometric.
	 \end{enumeratetheorem}
\end{theorem}
\begin{proof}
    \begin{enumerateproof}
        \item
            Let $y' \in Y'$.  Plugging it into the adjoint operator, we get
            $T' y' = y' \chain T$ with signature $X \rightarrow Y \rightarrow \thefield$.
            We can now see that $\im T' \subset X'$.

            \noindent Let $y'_1, y'_2 \in Y', \alpha \in \thefield$. We then prove linearity:
            \begin{align*}
                T'(\alpha y'_1 + y'_2) 
                &= (\alpha y'_1 + y'_2) \chain T                \explainstep{apply def. of adjoint operator} \\
                &= \alpha y'_1 \chain T + y'_2 \chain T         \explainstep{expand expression} \\
                &= \alpha T' y'_1 + T' y'_2                     \explainstep{apply def. of adjoint operator}
            \end{align*}

            \noindent Let $y' \in Y'$. We then prove boundedness of the operator norm:
            \begin{align*}
                \norm{X'}{T' y'}
                &= \norm{X'}{y' \chain T}                           \explainstep{apply def. of adjoint operator} \\
                &\le \norm{Y'}{y'}\norm{\contoperators{X}{Y}}{T}    \explainstep{apply def. of op. norm} \\
                &\defeq C \norm{Y'}{y'}                             \explainstep{def. the constant}
            \end{align*}

        \item
            Let $T_1, T_2 \in \contoperators{X}{Y}, y' \in Y', x \in X, \alpha \in \thefield$. We first prove linearity:
            \begin{align*}
                (\alpha T_1 + T_2)'(y')(x)
                &= y'(\alpha T_1x + T_2x)                           \explainstep{apply def. of adjoint operator} \\
                &= y'\parens{\alpha T_1x + T_2x}                    \explainstep{pull $x$ into the eq.} \\
                &= \alpha y'(T_1x) + y'(T_2x)                       \explainstep{y' is linear}
            \end{align*}

            \noindent We then prove isometry:
            \begin{align*}
                \norm{}{T}
                &= \sup_{\norm{X}{x} \le 1}\norm{Y}{Tx}                                 \explainstep{use supremum char. of op. norm} \\
                &= \sup_{\norm{X}{x} \le 1}\sup_{\norm{}{y'} \le 1}{\abs{y'(Tx)}}       \explainstep{apply theorem of Hahn-Banach\ \ref{theorem:hahn-banach-op-norm}} \\
                &= \sup_{\norm{}{y'} \le 1}\sup_{\norm{X}{x} \le 1}{\abs{y'(Tx)}}       \explainstep{supremum order can be switched} \\
                &= \sup_{\norm{}{y'} \le 1}{\norm{}{T' y'}}                             \explainstep{apply def. of adjoint operator and op. norm} \\
                &= \norm{}{T'}                                                          \explainstep{apply def. of op. norm}
            \end{align*}

    \end{enumerateproof}
\end{proof}

\begin{example}
    % TODO
\end{example}

\newcommand{\lp}{\ensuremath{l_p}\xspace}
\newcommand{\pconj}{\ensuremath{p^*}\xspace}
\newcommand{\lpconj}{\ensuremath{l_\pconj}\xspace}
\begin{example}
    Let $p \in (1, \infty)$ with $p \neq 2$. This makes \lp a Banach space according to the lecture,
    but not a Hilbert space as the parallelogram rule is not satisfied.
    We know that the dual space of \lp is isometrically isomorph to \lpconj
    where \pconj is the \highlight{Hölder conjugate} with $1/p + 1/p^* = 1$. \\
    As a reminder, the general idea of the proof of $\lp' \isomorphto \lpconj$ goes as follows:
    \begin{itemize}
        \item Define the isometric isomorphism as \[T: \lpconj \rightarrow \lp', s \mapsto (x \mapsto \sum_{k \in \naturals}{x_k s_k})\]
        \item Verify $(Ts)x$ converges as \[\abs{(Ts)x} \le \norm{p}{x}\norm{\pconj}{s} < \infty\] and absolute convergence implies convergence.
        \item Verify $T$ is injective using the linearity.
        \item Verify $T$ is surjective and isometric through todo (the annoying part).
    \end{itemize}

    \noindent To illustrate the adjoint operator, we now work through an example.
    Consider the \highlight{left shift} operator \[T: \lp \rightarrow \lp, \sequence{k}{x_k} \mapsto \sequence{k}{x_{k+1}}\]
    It is well-defined since \[\sum_{k \in \naturals}\abs{x_{k+1}}^p \le \sum_{k \in \naturals}\abs{x_{k}}^p < \infty\]

    \noindent We can now compute the adjoint operator $T'$:
    \begin{itemize}
        \item 
            The adjoint $T'$ must have the signature $\lp' \isomorphto \lpconj \rightarrow \lp' \isomorphto \lpconj$.

        \item 
            Let $y' \in \lp' \isomorphto \lpconj$.
            Then we can write $y': \lp \rightarrow \thefield, x \mapsto \sum_{k \in \naturals}{x_k s_k}$ with $s \in \lpconj$. \\
            Now for $x \in \lp$ we have
            \begin{align*}
                (T' y')(x)
                &= (y' \chain T)(x) = y'(Tx)                \explainstep{apply def. of $T'$} \\
                &= y'(\sequence{k}{x_{k+1}})                \explainstep{apply def. of $T$} \\
                &= \sum_{k \in \naturals}{x_{k+1} s_k}      \explainstep{apply def. of $y'$} \\
                &= \sum_{k \in \naturals}{x_{k} s'_k}       \explainstep{with $s'_1 = 0$ and $s'_k = s_{k-1}$ for $k > 1$}
            \end{align*}

            \noindent This tells us that the adjoint operator $T'$ acts as a \highlight{right shift} (up to isomorphism):
            \[T': \lpconj \rightarrow \lpconj, \sequence{k}{s_k} \mapsto (0, s_1, s_2, \dots)\]

    \end{itemize}
\end{example}

\begin{theorem}
    Let $X, Y, Z$ be normed \thefield\ vector spaces. \\
    Then the adjoint operator reverses composition:
    \[\forall T \in \contoperators{X}{Y}, \forall S \in \contoperators{Y}{Z}: (S \chain T)' = T' \chain S'\]
\end{theorem}
\begin{proof}
    Let $T \in \contoperators{X}{Y}$ and $S \in \contoperators{Y}{Z}$. \\
    We know $ST = S \chain T$ is still a linear, bounded operator from $X$ to $Z$.
    So $(ST)'$ is well-defined. \\
    Let $z' \in Z' = \contoperators{Z}{\thefield}$ and 
    set $y' = z' \chain S \in \dualspace{Y}$. We have
    \begin{align*}
        (ST)'(z') &= z' \chain (ST)     \explainstep{apply def. of adjoint operator} \\
        &= (z' \chain S) \chain T       \explainstep{write out chain explicitly} \\
        &= y' \chain T                  \explainstep{subst. y'} \\
        &= T' y'                        \explainstep{apply def. of adjoint operator} \\
        &= T' (z' \chain S)             \explainstep{subst. y'} \\
        &= T' S' z'                     \explainstep{apply def. of adjoint operator}
    \end{align*}
    So in total $(ST)' = T' S'$.
\end{proof}

\begin{example}
    % TODO: Add the example with i_X and i_Y. (Lemma III.4.3)
    % TODO: Or revisit the lp example again?
\end{example}

\subsection{The dual space of the dual space}

\noindent For starters, we need to recall some concepts from the lecture.

\begin{definition}
    Let $X$ be a normed \thefield\ vector space. 
    \begin{romanenum}
        \item $X''$ is called the bidual space.
        \item Let $J_X: X \rightarrow X'', p \mapsto (x' \mapsto x'(p))$.
            $J_X$ is called the \highlight{canonical embedding} from $X$ into $X''$.
    \end{romanenum} 
    So $J_X$ returns a function that evaluates dual space elements at $p \in X$.
\end{definition}

\noindent Figure [todo] illustrates why the bidual space in conjunction with the adjoint operator is interesting.
TODO

\newcommand{\canonicalembinv}[1]{\ensuremath{\inv{J_#1|_{J_#1(#1)}}}}
\begin{theorem}\label{theorem:bidual-adjunct-embedding}
    Let $(X, \norm{X}{\cdot})$ and $(Y, \norm{Y}{\cdot})$ be normed \thefield\ vector spaces.
    Let $T \in \contoperators{X}{Y}$ be a bounded linear operator.
    Then we have: $J_Y \chain T = T'' \chain J_X$. \\
    \comment{Or equivalently: $T''|_{J_X(X)} = J_Y \chain T \chain \canonicalembinv{X}$.} \\
    \comment{Or equivalently: $T = \canonicalembinv{Y} \chain T'' \chain J_X$.}
\end{theorem}
\begin{proof}
    Before the proof, we can avoid confusion by typing out what $T'$ and $T''$ evaluate to:
    \begin{itemize}
        \item $T': Y' \rightarrow X', y' \mapsto y' \chain T$ is the adjoint operator.
        \item $T'': X'' \rightarrow Y'', x'' \mapsto x'' \chain T'$ is the biadjoint operator.
    \end{itemize}

    \noindent Now first of all, we need to check that the signatures of both sides of the equation match:
    \begin{itemize}
        \item $J_Y: Y \rightarrow Y''$   and $T: X \rightarrow Y$     means $(J_Y \chain T): X \rightarrow Y''$.
        \item $T'': X'' \rightarrow Y''$ and $J_X: X \rightarrow X''$ means $(T'' \chain J_X): X \rightarrow Y''$.
    \end{itemize}

    \noindent Finally, for $p \in X$ and $y' \in Y'$ we have
    \begin{align*}
        \parens{(J_Y \chain T)(p)}(y')
        &= J_Y(Tp)(y') = y'(Tp)                 \explainstep{subst. $p$ in and apply def. of $J_Y$} \\
        &= (y'T)(p)                             \explainstep{use associativity} \\
        &= (T'y')(p)                            \explainstep{apply def. of $T'$} \\
        &= (x' \mapsto x'(p))(T'y')             \explainstep{pull out subst. function} \\
        &= J_X(p)(T'y')                         \explainstep{recognize this is just $J_X$} \\
        &= T''(J_X(p))(y')                      \explainstep{apply def. use $T''$} \\
        &= \parens{(T'' \chain J_X)(p)}(y')   \explainstep{use \chain\ notation}
    \end{align*}
\end{proof}

\noindent Secondly, we can answer when a continuous operator between $Y'$ and $X'$ is an adjoint operator.
We need to revise an important corollary from the lecture first.

\begin{revision}\label{theorem:embedding-isomorphism}
    Let $(X, \norm{}{\cdot})$ be a normed \thefield\ vector space. Then we have
    \begin{romanenum}
        \item The canonical embedding $J_X$ is an isometric injective function.
        \item $J: X \rightarrow J_X(X), x \mapsto J_X(x)$ is an isometric isomorphism.
        \item \canonicalembinv{X} is a bounded, linear operator.
    \end{romanenum}
\end{revision}
\begin{proofidea}
    \begin{romanprooflist}
        \item \refertolecture
        \item Follows by definition of $J$ and from ''i)''.
        \item
            Since $J: X \rightarrow J_X(X), x \mapsto J_X(x)$ is an isometric isomorphism, we have
            \begin{itemize}
                \item \canonicalembinv{X} inherits linearity from $J$.
                \item $\norm{}{J} = 1$ and therefore $\norm{}{\canonicalembinv{X}} = 1$.
                    This means \canonicalembinv{X} is bounded.
                    % TODO: Explain this better!
            \end{itemize}

    \end{romanprooflist}
\end{proofidea}

\begin{theorem}
    Let $S \in \contoperators{Y'}{X'}$ be a continuous, linear operator. \\
    Then we have $\exists T \in \contoperators{X}{Y}: T' = S \iff S'(J_X(X)) \subset J_Y(Y)$.
\end{theorem}
\begin{proof}
    \leftrightproof We have 
    \begin{align*}
        S'(J_X(X))
        &= T''(J_X(X))      \explainstep{substitute in $S = T'$} \\
        &= J_Y(T(X))        \explainstep{use theorem\ \ref{theorem:bidual-adjunct-embedding}} \\
        &\subset J_Y(Y)     \explainstep{use $T(X) \subset Y$}
    \end{align*}

    \noindent \rightleftproof 
    We have $S'(J_X(X)) \subset J_Y(Y)$ and revision\ \ref{theorem:embedding-isomorphism} gives us $J_Y(Y) \isomorphto Y$. \\
    So for $x \in X$ and $y''_x = S'(J_X(x))$ there is a (unique) $y_x \in Y$ with $y''_x = J_Y(y_x)$. \\
    Define $T: X \rightarrow Y, x \mapsto y_x$. We know $T$ exists (and is unique) due to the previous argument.

    \noindent We know $T$ is linear and continuous, as $T = y_\cdot = J_Y^{-1} \chain S' \chain J_X$ 
    and all elements in the chain are bounded, linear operators.
    Let $y' \in Y'$ and $x \in X$. Lastly, we need to prove that $S = T'$:
    \begin{align*}
        (Sy')(x)
        &= J_X(x)(S y')                                             \explainstep{express using $J_X$} \\
        &= (S' J_X(x))(y') = (S' \chain J_X)(x)(y')                 \explainstep{apply def. of ajoint op.} \\
        &= J_Y((\canonicalembinv{Y} \chain S' \chain J_X)(x))(y')   \explainstep{use $J_Y \chain \canonicalembinv{Y} = \id$} \\
        &= y'((\canonicalembinv{Y} \chain S' \chain J_X)(x))        \explainstep{evaluate $J_Y$ at the vector} \\
        &= y'(Tx) = (y' \chain T)(x)                                \explainstep{subst. in $T = J_Y^{-1} \chain S' \chain J_X$} \\
        &= (T' y')(x)                                               \explainstep{apply def. of adjoint op.}
    \end{align*}

\end{proof}

\begin{example}\label{example:operator-is-not-adjoint}
    We have already seen an adjunct operator.
    Using the last theorem, we can find an example for an operator 
    $S \in \contoperators{Y'}{X'}$ that is not an adjoint operator \\
    (i.e.\ with $\nexists T \in \contoperators{X}{Y}: T' = S$).

    \noindent TODO.
    % TODO: A not-adjunct operator :) we can use later!
\end{example}

% TODO: We should probably merge the example with the theorem.
\begin{theorem}
    Lastly, we get one more property of the adjoint:
    $T \mapsto T'$ is not always surjective. \\
    \note{This is not the case with $\cdot^H$ e.g.\ between $\reals^n \rightarrow \reals^n$.}
\end{theorem}
\begin{proof}
    The operator and the adjoint operator have the following signatures:
    \begin{itemize}
        \item $T: X \rightarrow Y$
        \item $T': Y' = \dualspace{Y} \rightarrow X' = \dualspace{X}$
    \end{itemize}
    So the ``not always'' refers to a particular choice of $X$ and $Y$ we need to find.
    Fortunately in example \ref{example:operator-is-not-adjoint} we already found a counterexample!
\end{proof}


% \subsection{Examples for Adjoint Operators on Banach Spaces}

% TODO: Examples for dual operators in l_p, L_p and some identity into the dual-dual space.
% TODO: Make sure to find out which is which.

% TODO: Maybe you can add an example from optimisation?

\subsection{Compact (adjoint) operators}

\noindent For starters, we need to recall and review some concepts from the lecture.

\begin{definition}
    The following definitions are revisions from the lecture: \\
    Let $X, Y$ be normed \thefield\ vector spaces.
    \begin{enumeratetheorem}
        \item $M \subset X$ is relatively compact, if \[\forall (x_n)_{n \in \naturals} \subset M: (x_n)_{n \in \naturals} \text{ has a converging subsequence in } X\]
        \item $T \in \contoperators{X}{Y}$ is compact, if $T\parens{\closedunitball{X}}$ is relatively compact.
        \item The rank of T is $\rk T = \dim{T(X)}$.
    \end{enumeratetheorem}
\end{definition}

% Explanation of the significance of these definitions:
% TODO: Add examples to this section. The motivation is still not clear enough.
\noindent As a reminder, in metric spaces a subset is compact if and only if all sequences contain a converging subsequence in that set.
So relatively compact relaxes the requirement that the set must be closed.
An important property of the closed unit ball in $\reals^n$ or $\complexes$ is that it is compact.
Therefore, a linear operator between $\thefield^n \rightarrow \thefield^m$ always maps the closed unit ball to a compact set.
But the domain might not be a finite-dimensional normed space and the unit ball of the domain might not be compact either.

Even when we only know that a bounded operator has finite rank, we can see that it is always compact:
In finite dimensions, all norms are equivalent. So $(X, \norm{}{\cdot})$ and $(X, \norm{1}{\cdot})$ have the same open sets.
In finite dimensions, linear algebra tells us that $(X, \norm{1}{\cdot})$ and $(\reals^{\dim{X}}, \norm{1}{\cdot})$ have the same open sets.
So all finite-dimensional normed spaces are topologically equivalent to some space $\reals^n$.
In $\reals^n$, the theorem of Heine-Borel tells us that all bounded subsets are relatively compact.

Since relatively compact is a topological property, the theorem transfers to $(X, \norm{}{\cdot})$.
Finally, a finite rank causes the bounded image of the operator to be finite-dimensional, which then causes it to be relatively compact.
This is a very strong statement, considering that the domain unit ball might not be compact at all.

This finally breaks down when the rank is infinite.
So a compact operator simply ensures that we still can enjoy this finite-dimensional behavior between general Banach spaces.

\begin{definition}
    The following definitions are critical for the theorem of \highlight{Arzelà-Ascoli}: \\
    Let $X, Y$ be metric spaces and $M \subset \braces{f: X \rightarrow Y}$.
    \begin{enumeratetheorem}
        \item $M$ is uniformly equicontinuous, if
            \[ \forall \epsilon > 0, \exists \delta > 0, \forall T \in M, \forall x, y \in X: d_X(x, y) < \delta \implies d_Y(Tx, Ty) < \epsilon \]
        \item $M$ is pointwise bounded, if 
            \[ \forall x \in X: \braces{f(x) \mid f \in M} \text{ is bounded} \]
    \end{enumeratetheorem}
\end{definition}

\noindent The theorem of \highlight{Arzelà-Ascoli} from the lecture gives us a characterisation of relative compactness 
for the set of continuous functions on a compact metric space using the two previous concepts.
This is interesting, because uniform equicontinuity and pointwise boundedness are (relatively) simple properties 
of the functions \highlight{evaluations} rather than the functions themselves. They can be easily verified to be true.
Note that $C(D) = \braces{f: D \rightarrow \thefield \text{ continuous}}$.

\begin{revision}[Arzelà-Ascoli]
    Let $D$ be a compact metric space and $M \subset C(D)$ with the supremum norm.
    Then we have $M$ is uniformly equicontinuous and pointwise bounded implies $M$ is relatively compact.
\end{revision}
\begin{proofidea}
    \begin{itemize}
        \item $D$ is separable, i.e. $\exists D_0 \subset D: \overline{D_0} = D$. Simply set $D_0 = \bigcup_{n \in \naturals}{D_n}$ where $D_n$ is a finite $\frac{1}{n}$-covers of $D$.
        \item For all $x \in D$ we can use the pointwise boundedness to invoke Bolzano-Weierstrass to find converging subsequences $(f_{n(x, k)}(x))_{k \in \naturals} \subset \thefield$.
        \item Specifically, we have $\forall x \in D_0: (f_{n(x, k)}(x))_{k \in \naturals} \text{ converges}$.
        \item Using the commonly used diagonal argument from the lecture, we can find a unified subsequence with no dependence on $x$:
            $\forall x \in D_0: (f_{n(k)}(x))_{k \in \naturals} \text{ converges}$.
        \item We already have $f_{n(k)}|_{D_0} \convergesto: f|_{D_0}$.  And it seems sensible to assume that $(f_{n(k)})_{k \in \naturals}$ is a convergent subsequence.
            % TODO: \convergesto: is bad notation
            But how can you extend this to the entire set $D$?
        \item For each $x \in D$, we can choose an arbitrarily close $x_0 \in D_0$. Using two triangle inequalities, uniform equicontinuity allows us to extend the result to $D$. \\
            As $C(D)$ is complete, the Cauchy sequence converges.
    \end{itemize}
\end{proofidea}

\noindent Using the revision, we can now prove the theorem of Schauder and then trace the arguments through both proofs to get a better understanding of the ideas.

\begin{theorem}[Schauder]
    Let $X, Y$ be Banach spaces and $T: X \rightarrow Y$ be a bounded, linear operator.
    Then we have $T$ is compact if and only if $T'$ is compact.
\end{theorem}
\begin{proof}
    \leftrightproof
    Let $(y'_n)_{n \in \naturals} \subset Y' = \dualspace{Y} \subset C(Y)$ be bounded. \\
    Our goal is to show that there is a convergent subsequence in \sequence{n}{T' y'_n}
    with respect to $(X', \norm{}{\cdot})$ where $\norm{}{\cdot}$ is the operator norm.
    For all $n \in \naturals$ set $f_n = y'_n|_{T\parens{\closedunitball{X}}}$.

    \noindent We have
    \begin{align*}
        \norm{}{T' y'_n - T' y'_m} 
        &= \norm{}{y'_n \chain T - y'_m \chain T}                                   \explainstep{apply def. of adjoint op.} \\
        &= \sup_{\norm{X}{x} \le 1}\abs{((y'_n \chain T) - (y'_m \chain T))(x)}     \explainstep{use supremum char. of the op. norm} \\
        &= \sup_{\norm{X}{x} \le 1}\abs{((f_n \chain T) - (f_m \chain T))(x)}       \explainstep{subst. in $f_n$ and $f_m$} \\
        &= \sup_{d \in T\parens{\closedunitball{X}}}\abs{f_n(d) - f_m(d)}           \explainstep{subst. in $f_n$ and $f_m$} \\
    \end{align*}
    We know \sequence{n}{T' y'_n} converges if and only if it is a Cauchy sequence $(\dualspace{X}, \norm{}{\cdot})$.
    Therefore the convergence of \sequence{n}{T' y'_n} in the operator norm is only dependent on the behaviour
    of \sequence{n}{f_n} on $\overline{T\parens{\closedunitball{X}}}$.

    \noindent We now set \[ D = \overline{T\parens{\closedunitball{X}}} \]
    and pack the sequence into \[ M \defeq \braces{f_n \mid n \in \naturals} \]
    and examine them for
    \begin{itemize}
        \item $D$ is compact:
            We have
            \begin{itemize}
                \item \closedunitball{X} is bounded.
                \item $T$ is a compact operator.
            \end{itemize}
            So $T\parens{\closedunitball{X}}$ is relatively compact and $D$ is compact.

        \item $M$ is pointwise bounded: For $x \in \closedunitball{X}$ and $n \in \naturals$ we have
            \begin{align*}
                \abs{f_n(Tx)}
                &= \abs{y'_n(Tx)} \defeq \abs{y'_n(d)}      \explainstep{apply def. of $f_n$ and define $d$} \\
                &\le C_1 \norm{Y}{d} \le C_1 C_2            \explainstep{apply op. norm ineq. and D bounded means $\norm{Y}{d} \le: C_2$}
            \end{align*}
            For $d \in D$ we can choose $\sequence{k}{x_k} \subset \closedunitball{X}$ with
            $T x_k \convergesto d$ and $f_n(T x_k) \le C_1 C_2$. \\
            Using continuity we get $\abs{f_n(d)} \le C_1 C_2$.

        \item $M$ is uniformly equicontinuous: 
            For $n \in \naturals, \epsilon > 0, \delta = \epsilon / C_1, \forall d_1, d_2 \in D, \norm{Y}{d_1 - d_2} < \delta$ we have
            \begin{align*}
                \abs{f_n(d_1) - f_n(d_2)}
                &\le \norm{}{y'_n} \cdot \norm{Y}{d_1 - d_2}        \explainstep{factor out and apply op. norm ineq.} \\
                &\le C_1 \norm{Y}{d_1 - d_2}                        \explainstep{use the upper bound $\norm{}{y'_n} \le C_1$} \\
                &< C_1 \delta = \epsilon                            \explainstep{substitute in $\delta$}
            \end{align*}

    \end{itemize}

    \noindent The theorem of Arzelà-Ascoli now tells us that $M$ is relatively compact.
    So every sequence in $M$ has a convergent subsequence.
    In particular for the convergent subsequence \sequence{k}{f_{n(k)}} we have
    \begin{align*}
        \sequence{k}{f_{n(k)}}
        &= \sequence{k}{y'_n \chain T}      \explainstep{apply def. of $f_{n(k)}$} \\
        &= \sequence{k}{T' y'_{n(k)}}       \explainstep{apply def. of adjoint operator}
    \end{align*}

    \noindent So finally, \sequence{k}{T' y'_{n(k)}} is a convergent subsequence of \sequence{k}{T' y'_k}. \\

    \noindent \rightleftproof
    The other proof direction tells us that \[ T \text{ compact } \implies T' \text{ compact} \]
    So in extension this also yields \[ T' \text{ compact } \implies T'' \text{ compact} \]
    Corollary\ \ref{corollary:bidual-adjunct-embedding-ext} tells us that \[ T = \canonicalembinv{Y} \chain T'' \chain J_X \]
    Lastly, the operator $T''$ is compact, 
    theorem\ \ref{revision:embedding-isomorphism} tells us $J_X$ and \canonicalembinv{Y} are bounded, linear operators
    and therefore $T$ is a composition of bounded, linear operators. \\
    Using the revision theorem\ \ref{revision:general-revision} we can conclude that $T$ is compact.

\end{proof}

\begin{example}
    The lecture\ \cite{krieg_functional_2025} states that the following integral operator is compact:
    \[ T: C[0,1] \to C[0,1], T f(x) = \int_0^1{f(t) \,dx} \]
    So $T': C[0,1]' \to C[0,1]'$ is compact too.
    Evaluating $C[0,1]'$ is beyond the scope of this paper.
\end{example}

%%%%%%%%%%%%%%%%%%%%%%%%%%%%%%%%%%%%%%%%%%%%%%%%%%%%%%%%%%%%%%%%%%%%%%%%%%%%%%%
\subsection{The rank-nullity theorem for operators}
%%%%%%%%%%%%%%%%%%%%%%%%%%%%%%%%%%%%%%%%%%%%%%%%%%%%%%%%%%%%%%%%%%%%%%%%%%%%%%%

\newcommand{\anihv}{\ensuremath{V_\perp}}

\begin{frame}{Definition: Annihilator}
    \begin{definition}
        Let $(X, \norm{}{\cdot})$ be a normed \thefield\ vector space and $V \subset X'$.
        Then we define the annihilator of $V$ in $X$ as
        \[ \anihv = \braces{x \in X \mid \forall x' \in V: x'(x) = 0} \]
        \sidenote{
            The annihilator is the set of linear, bounded functionals
            that "see" exactly the opposite of $V$
            and are "blind" to $V$.
        }
    \end{definition}
\end{frame}

\begin{frame}{Revision: Hahn-Banach Corollary}
    \begin{block}{Revision}
        Let $(X, \norm{}{\cdot})$ be a normed \thefield\ vector space,
        $U \subset X$ a closed subspace and $x \in X \setminus U$.
        Then we have
        \[ \exists x' \in X': x'|_U = 0 \logicaland x'(x) \neq 0 \]
    \end{block}
\end{frame}

\begin{frame}{Theorem: Annihilator is Closed Linear Subspace}
    \begin{theorem}
        Let $(X, \norm{}{\cdot})$ be a normed \thefield\ vector space and $V \subset X'$.
        Then we have
        \[ \anihv \subset X \text{ is a closed linear subspace} \]
    \end{theorem}
    \begin{proof}
        We have
        $ \anihv = \bigcap_{x' \in V}{\inv{(x')}(0)} $
        As an intersection of closed sets, \anihv must be closed.
    \end{proof}
\end{frame}

\newcommand{\kerdualperp}{\ensuremath{(\ker T')_\perp}\xspace}

\begin{frame}{Theorem: Rank-Nullity Generalized}
    \begin{theorem}
        Let $T \in \contoperators{X}{Y}$ be a bounded, linear operator.
        Then we have \[ \overline{\im T} = \kerdualperp \]
        \sidenote{
            In linear algebra lectures this is proven for finite-dimensional vector spaces
            (see Satz 6.1.5\ \cite{werner_funktionalanalysis_2018})
        }
    \end{theorem}
\end{frame}

\begin{frame}{Theorem: Rank-Nullity Generalized -- Proof ($\subset$)}
    \begin{block}{Proof}
        \leftrightinclusion
        Let $Tx \in \im T$ with $x \in X$ and $y' \in \ker T'$. \\
        We first prove $Tx \in \kerdualperp$:
        \begin{align*}
            y'(Tx) &= (y' \chain T)(x)  \explainstep{associativity and use \chain\ notation} \\
            &= (T' y')(x)               \explainstep{apply adjoint op. definition} \\
            &= 0(x) = 0                 \explainstep{apply $y' \in \ker T'$}
        \end{align*}
        Since this holds for all choices of $Tx$ we get $ \im T \subset \kerdualperp $
        Since \kerdualperp is closed, we also get $ \overline{\im T} \subset \kerdualperp $
    \end{block}
\end{frame}

\begin{frame}{Theorem: Rank-Nullity Generalized -- Proof ($\supset$)}
    \begin{proof}
        \rightleftinclusion
        We can prove the contraposition
        $ \parens{Y \setminus \overline{\im T}} \subset \parens{Y \setminus \kerdualperp} $
        Set $U = \overline{\im T}$ and let $y \in Y \setminus U$.
        We know $U$ that a closed linear subspace.
        The corollary of the theorem of Hahn-Banach tells us that
        $ \exists y' \in Y' : y'|_U = 0 \logicaland y'(y) \neq 0 $
        Since $\ker T' \subset Y'$
        and $\forall y' \in \ker T' : y'(y) \neq 0$
        we get $ y \in Y \setminus \kerdualperp $
    \end{proof}
\end{frame}

\begin{frame}{Corollary: Operator Solutions}
    \begin{corollary}
        Let $T \in \contoperators{X}{Y}$ be a linear, continuous operator with $\im T$ closed. Then we have
        \[ y \in \im T \text{ if and only if } \forall y' \in Y': T'y' = 0 \implies y'(y) = 0 \]
    \end{corollary}
    \begin{proof}
        We have
        \begin{align*}
            y \in \im T
            &= \overline{\im T}                                     \explainstep{$\im T$ is closed} \\
            &= \kerdualperp                                         \explainstep{apply the theorem} \\
            \iff &\forall y' \in \ker T' : y'(y) = 0                \explainstep{apply def. of annihilator} \\
            \iff &\forall y' \in Y': T'y' = 0 \implies y'(y) = 0    \explainstep{write in equivalent way}
        \end{align*}
    \end{proof}
\end{frame}

\begin{frame}{Example: Left Shift}
    \begin{example}
        Consider the left shift operator $T: \lp \to \lp$ from earlier.
        Its adjoint $T': \lpconj \to \lpconj$ is the right shift operator:
        $ T'(s_1, s_2, \dots) = (0, s_1, s_2, \dots) $

        To apply the theorem, we first find $\ker T'$.
        If $T's = 0$, then $(0, s_1, s_2, \dots) = (0, 0, 0, \dots)$, which implies $s = 0$.
        Thus, $\ker T' = \{0\}$.

        The theorem states that $\overline{\im T} = (\ker T')_\perp = \{0\}_\perp = \lp$.
        This is consistent with the fact that the left shift is surjective ($\im T = \lp$).
    \end{example}
\end{frame}

\begin{frame}{Example: Right Shift}
    \begin{example}
        Consider the right shift operator $T: \lp \to \lp, T(x_1, x_2, \dots) = (0, x_1, x_2, \dots)$.
        Its image is the closed subspace $ \im T = \{ y \in \lp \mid y_1 = 0 \} $.

        The adjoint $T': \lpconj \to \lpconj$ is the left shift, $T'(s_1, s_2, s_3, \dots) = (s_2, s_3, \dots)$.
        Thus $\ker T' = \{ (s_1, 0, 0, \dots) \mid s_1 \in \thefield \} = \text{span}\{e_1'\}$.

        By the corollary, $y \in \im T$ iff it is annihilated by every functional in $\ker T'$.
        For any $s \in \ker T'$, the condition $s(y) = 0$ becomes:
        $ \sum_{k=1}^\infty y_k s_k = y_1 s_1 = 0 $

        Since this must hold for all $s_1 \in \thefield$, we must have $y_1 = 0$.
        This matches $\im T$, verifying the theorem.
    \end{example}
\end{frame}


%%%%%%%%%%%%%%%%%%%%%%%%%%%%%%%%%%%%%%%%%%%%%%%%%%%%%%%%%%%%%%%%%%%%%%%%%%%%%%%
\section{References}
%%%%%%%%%%%%%%%%%%%%%%%%%%%%%%%%%%%%%%%%%%%%%%%%%%%%%%%%%%%%%%%%%%%%%%%%%%%%%%%

\begin{frame}{References}
    \printbibliography
\end{frame}

\begin{frame}{Summary}
    \begin{itemize}
        \item Defined the \highlight{adjoint operator} $T': Y' \to X'$ for bounded linear operators
        \item Showed $T \mapsto T'$ is \highlight{linear and isometric}
        \item Proved the adjoint \highlight{reverses composition}: $(ST)' = T'S'$
        \item Characterized when an operator $S: Y' \to X'$ is an adjoint via the \highlight{canonical embedding}
        \item Proved \highlight{Schauder's theorem}: $T$ compact $\iff$ $T'$ compact
        \item Generalized the \highlight{rank-nullity theorem}: $\overline{\im T} = (\ker T')_\perp$
    \end{itemize}
\end{frame}

\end{document}
