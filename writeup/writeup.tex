\documentclass{article}

\usepackage[english]{babel}
\usepackage[a4paper,top=2cm,bottom=2cm,left=3cm,right=3cm,marginparwidth=1.75cm]{geometry}
\usepackage{amsmath, amsfonts, graphicx, enumitem, xcolor}
\usepackage[colorlinks=true, allcolors=blue]{hyperref}
\usepackage{dsfont, mathtools, amsthm, diagbox, caption, subcaption}
\usepackage[
    backend=bibtex,
    style=alphabetic,
]{biblatex}
\addbibresource{../bibliography.bib} %Imports bibliography file

\title{Dual Operators}
\author{Erik Stern}

\newcommand{\defeq}{\vcentcolon=}
\newcommand{\eqdef}{=\vcentcolon}
\renewcommand{\implies}{\Rightarrow}
\newcommand{\normedspace}[1]{(#1, \|\cdot\|_{#1})}
\newcommand{\highlight}[1]{\textcolor{blue}{#1}}
\newcommand{\norm}[2]{\|#2\|_{#1}}
\newcommand{\contoperators}[2]{\ensuremath{\mathcal{L}\left(#1, #2\right)}}
\newcommand{\thefield}{\ensuremath{\mathbb{K}}}
\newcommand{\chain}{\circ}
\newcommand{\explainstep}[1]{&& \textcolor{darkgray}{\text{(#1)}}}
\newcommand{\comment}[1]{\textcolor{darkgray}{\text{#1}}}
\newcommand{\leftrightproof}{\glqq\ensuremath{\Rightarrow}\grqq:\quad}
\newcommand{\rightleftproof}{\glqq\ensuremath{\Leftarrow}\grqq:\quad}
\newcommand{\logicaland}{\ensuremath{\wedge}}

% ===== BRACKETS =====
\newcommand{\parens}[1]{\left(#1\right)}
\newcommand{\brackets}[1]{\left[#1\right]}
\newcommand{\braces}[1]{\left\{#1\right\}}

% ===== SETS ======
\newcommand{\reals}{\ensuremath{\mathbb{R}}}
\newcommand{\complexes}{\ensuremath{\mathbb{C}}}
\newcommand{\naturals}{\ensuremath{\mathbb{N}}}

% ===== SPECIAL OPERATORS =====
\DeclareMathOperator{\rk}{rk}

% ===== ENUMERATIONS =====
\newlist{romanenum}{enumerate}{1}
\setlist[romanenum]{label=\roman*):, leftmargin=*, labelsep=0.5em, align=left}

\newlist{romanprooflist}{enumerate}{1}
\setlist[romanprooflist]{label="\roman*)":, leftmargin=*, labelsep=0.5em, align=left}


\newenvironment{proofidea}
{\begin{proof}[Proof Idea]}
{\end{proof}}



% ===== THEOREM STYLING =====
\theoremstyle{plain}
\newtheorem{theorem}{Theorem}
\newtheorem{revision}{Revision}
\newtheorem{corollary}{Corollary}
\theoremstyle{definition}
\newtheorem{definition}{Definition}
\newtheorem{example}{Example}
\newtheorem{remark}{Remark}



% ===== DOCUMENT ====
\begin{document}
\maketitle

\begin{abstract}
    This seminar paper provides a foundational overview of Dual-/Adjunct-Operators in functional analysis.
\end{abstract}

\tableofcontents

\section{Revision}

\begin{definition}
    The following basic definitions:
    \begin{romanenum}
        \item Banach Space
        \item Hilbert Space
    \end{romanenum}
\end{definition}

\begin{definition}
    Let $\normedspace{X}$ and $\normedspace{Y}$ be normed vector spaces on \reals or \complexes. \\
    Let $T: X \rightarrow Y$ be linear. \\
    $T$ is a \highlight{linear operator}.
    $T$ is \highlight{bounded}, if $\exists C > 0, \forall x \in X: \norm{Y}{Tx} \le C \norm{X}{x}$.
\end{definition}

\begin{definition}
    The (topological) dual space is defined as $X^{*} \defeq \contoperators{X}{\mathbb{R}}$.
\end{definition}

\begin{theorem}
    The topological dual space \contoperators{X}{Y} is a Banach space.
\end{theorem}
\begin{proofidea}
    
\end{proofidea}

\begin{theorem}
    For linear operators, continous and bounded are equivalent.
\end{theorem}
\begin{proofidea}
    TOOD
\end{proofidea}

\begin{definition}
    \begin{romanenum}
        \item 
        \item 
    \end{romanenum}
\end{definition}

\begin{theorem}
    \begin{romanenum}
        \item % l_p isomorphic to ...
            Let $p \in [1, \infty)$.
            Then 
        \item % L_p isomorphic to ...
    \end{romanenum}
\end{theorem}
\begin{proofidea}
    TOOD
\end{proofidea}


\begin{theorem}
\end{theorem}
\begin{proof}
    See functional analysis lecture of SS/25.
\end{proof}

% TODO: Prove that linear, bounded operators from X->Y and Y->Z can be chained together to be from X->Z linear, bounded.


% TODO: Sublinear Definition and Theorem of Hahn-Banach

% TODO: Why does Cauchy-Schwarz inequality work for operator norms on bounded, linear operators?


\newpage
\section{The Adjunct Operator}
\subsection{Definitions and Basic Properties}

\begin{remark}
    The terms \highlight{adjunct} and \highlight{dual} are often used interchangeably.
    We will standardize on \highlight{adjunct}, to avoid unnecessary confusion. \\
    We will abstract the field as $\thefield \in \{\reals, \complexes\}$.
\end{remark}

\begin{definition}
    Let $X, Y$ be metric spaces and $T \in \contoperators{X}{Y}$. \\
	 Then $T': Y' \rightarrow X', y' \mapsto y' \chain T$ is called the adjunct operator.
    From now on, we will implicitly refer to the metric spaces $X, Y$ and the topological dual spaces $X', Y'$ when talking about the dual operator $T'$.
\end{definition}

\begin{remark}
    In comparison to the operators we have previously worked with, the adjunct operator takes and outputs linear, bounded operators.
	 It's one more level of abstraction removed from \thefield. \\
	 For $y' \in Y' = \contoperators{Y}{\thefield}$, the adjunct operator evaluates to $T' y' \in \contoperators{X}{\thefield}$.
\end{remark}


\begin{theorem}
    We can collect some basic properties of the dual operator:
    \begin{romanenum}
        \item $T' \in \contoperators{Y'}{X'}$, so $T'$ is linear and bounded. \\
            \comment{This implies $\forall y' \in \contoperators{Y'}{X'}: T'y' \in X'$}.
        \item $T \mapsto T'$ is linear and isometric.
        \item $T \mapsto T'$ is not always surjective.
	 \end{romanenum}
\end{theorem}
\begin{proof}
    \begin{romanprooflist}
        \item
            We first prove linearity: \\
            \begin{align*}
                T'(\alpha y'_1 + y'_2) 
                &= (\alpha y'_1 + y'_2) \chain T                \explainstep{apply def. of dual operator} \\
                &= \alpha y'_1 \chain T + y'_2 \chain T         \explainstep{expand expression} \\
                &= \alpha T' y'_1 + \beta T' y'_2               \explainstep{apply def. of dual operator} \\
                &\text{for all } y'_1, y'_2 \in Y' \text{ and } \alpha \in \thefield
            \end{align*}

            We then prove boundedness of the operator norm: \\
            \begin{align*}
                \norm{X'}{T' y'}
                &= \norm{X'}{y' \chain T}                           \explainstep{apply def. of dual operator} \\
                &\le \norm{Y'}{y'}\norm{\contoperators{X}{Y}}{T}    \explainstep{Cauchy-Schwarz inequality} \\
                &\defeq C \norm{Y'}{y'}                             \explainstep{def. the constant} \\
                &\text{for all } y' \in Y'
            \end{align*}

        \item
            We first prove linearity: \\
            \begin{align*}
                (\alpha T_1 + T_2)'(y')
                &= y'\left(\alpha T_1 + T_2\right)              \explainstep{apply def. of dual operator} \\
                &= \alpha y'(T_1) + y'(T_2)                     \explainstep{y' is linear} \\
                &\text{for all } T_1, T_2 \in \contoperators{X}{Y} 
                \text{ and } y' \in Y' 
                \text{ and } \alpha \in \thefield
            \end{align*}

            We then prove isometry: \\
            \begin{align*}
                \norm{}{T} % TODO: This must be proven using Hahn-Banach.
            \end{align*}
      

        \item
            We construct a counterexample: \\


    \end{romanprooflist}
\end{proof}

\begin{example}
    % TODO
\end{example}

\begin{revision}[The Adjunct Operator reverses composition]
    Let $X, Y, Z$ be normed metric spaces. \\
    Then we have $\forall T \in \contoperators{X}{Y}, \forall S \in \contoperators{Y}{Z}: (S \chain T)' = T' \chain S'$.
\end{revision}
\begin{proof}
    Let $T \in \contoperators{X}{Y}$ and $S \in \contoperators{Y}{Z}$. \\
    We know $ST = S \chain T$ is still a linear, bounded operator from $X$ to $Z$.
    So $(ST)'$ is well-defined. \\
    Let $z' \in Z' = \contoperators{Z}{\thefield}$ and set $y' = z' \chain S \in \contoperators{Y}{\thefield}$. \\
    We can now evaluate the expression on $z'$:
    \begin{align*}
        (ST)'(z') &= z' \chain (ST)     \explainstep{apply def. of dual operator} \\
        &= (z' \chain S) \chain T       \explainstep{write out chain explicitly} \\
        &= y' \chain T                  \explainstep{subst. y'} \\
        &= T' y'                        \explainstep{apply def. of dual operator} \\
        &= T' (z' \chain S)             \explainstep{subst. y'} \\
        &= T' S' z'                     \explainstep{apply def. of dual operator}
    \end{align*}
    So in total, $(ST)' = T' S'$.
\end{proof}

\begin{example}
    % TODO: Add the example with i_X and i_Y. (Lemma III.4.3)
\end{example}

\newpage
\subsection{In Hilbert Spaces}
Risz-Frechet for Comparison between Hilbetraum-Adjunct and Banachraum-Adjunct.

\newpage
\subsection{Examples for Adjunct Operators on Banach Spaces}

\newpage
\subsection{Examples for Adjunct Operators on Hilbert Spaces}

\newpage
\subsection{Examples for Adjunct Operators in Finite Dimensional Hilbert Spaces}

% TODO: Examples for dual operators in l_p, L_p and some identity into the dual-dual space.
% TODO: Make sure to find out which is which.

% TODO: Maybe you can add an example from optimisation?

\newpage
\subsection{Relation to Compactness}

\begin{definition} % NOTE: Revision.
    The following definitions are revisions from the lecture: \\
    Let $X, Y$ be normed metric spaces.
    \begin{romanenum}
        \item $X$ is relatively compact, if $\forall (x_n)_{n \in \naturals} \subset X \text{ bounded}: (T x_n)_{n \in \naturals} \text{ has a converging subsequence}$.
        \item $T \in \naturals$ is compact, if $T B(0, \le 1, X) \text{ is relatively compact}$.
        \item The rank of T is $\rk T = \dim{T(X)}$.
    \end{romanenum}
\end{definition}

\begin{remark}
    % TODO: Explain the significance of these definitions.
\end{remark}

\begin{definition}
    The following definitions are critical for the theorem of \highlight{Arzelà-Ascoli}: \\
    Let $X, Y$ be metric spaces and $M \subset \braces{f: X \rightarrow Y}$.
    \begin{romanenum}
        \item $M$ is uniformly equicontinous, if \\
            $\forall \epsilon > 0, \exists \delta > 0, \forall f \in M, \forall x, y \in X: d_X(x, y) < \delta \implies d_Y(Tx, Ty)$.
        \item $M$ is pointwise bounded, if $\forall x \in X: \braces{f(x) \mid f \in M} \text{ is bounded}$.
    \end{romanenum}
\end{definition}

\begin{revision}[Arzelà-Ascoli]
    Let $D$ be a compact metric space and $\subset C(D) = \braces{f: D \rightarrow \thefield \text{ continous}}$. \\
    Then we have 
    \begin{align*}
        M \text{ relatively-compact} \iff M \text{ uniformly equicontinous} \logicaland M \text{ pointwise bounded}
    \end{align*}
\end{revision}
\begin{proofidea}
    TODO
\end{proofidea}

\begin{definition}
    Let $X$ be a normed vector space. 
    \begin{romanenum}
        \item $X''$ is called the bidual space.
        \item $i: X' \rightarrow \thefield$ % TODO: Define i_X and prove it's a bounded, linear operator
    \end{romanenum} 
\end{definition}

\begin{theorem}[Schauder]
    Let $X, Y$ be Banach spaces and $T: X \rightarrow Y$ a linear, continous operator.
    Then we have
    \begin{align*}
        T \text{ compact} \iff T' \text{ compact}
    \end{align*}
\end{theorem}
\begin{proof}
    \leftrightproof
    TODO \\
    \rightleftproof
    We have $T'$ is compact. Using the first part of the proof, we know $T''$ is compact. \\
    Since $T'' \chain i_X = i_Y \chain T$ we know $i_Y \chain T$ is compact. \\
    Since $Y \subset Y''$ is closed, we get $T$ compact. % TODO: Why????
\end{proof}

% IDK if this fits in at all????
% \begin{example}
%     % General example with compact integral operator (Script Example 3.35)
%     % Specify to bounded fourier transform.
%     % Somehow connect to dual topological space??? Idk if this is good. The punchline is maybe not ideal.
% \end{example}

% \begin{theorem}[Schauder]
% \end{theorem}

% \begin{theorem}[Schauder]
% \end{theorem}

\newpage
\subsection{Operator Equations}

% TODO: Theorem about solutions to Tx=y
% TODO: Example as to how it relates to Lagrange Multiplicators in Optimisation

\newpage
\printbibliography

% TODO: Add reminders everywhere. E.g. remind the reader of the definition of topological dual spaces when appropriate. We need to assume the reader is new to the subject.

% TODO: I like the analogy that compact operators are "almost finite-dimensional" because bounded sets are mapped to compact sets (modulu closedness).

\end{document}
