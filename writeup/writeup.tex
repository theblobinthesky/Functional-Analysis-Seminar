\documentclass{article}

\usepackage[english]{babel}
\usepackage[a4paper,top=2cm,bottom=2cm,left=3cm,right=3cm,marginparwidth=1.75cm]{geometry}
\usepackage{amsmath, amsfonts, amssymb, graphicx, enumitem, xcolor}
\usepackage[colorlinks=true, allcolors=blue]{hyperref}
\usepackage{dsfont, mathtools, amsthm, diagbox, caption, subcaption, xspace}
\usepackage[
    backend=bibtex,
    style=alphabetic,
]{biblatex}
\addbibresource{../bibliography.bib}

\title{Dual Operators}
\author{Erik Stern}

\newcommand{\defeq}{\vcentcolon=}
\newcommand{\eqdef}{=\vcentcolon}
\newcommand{\isomorphto}{\cong}
\renewcommand{\implies}{\Rightarrow}
\newcommand{\convergesto}{\longrightarrow}
\newcommand{\normedspace}[1]{(#1, \|\cdot\|_{#1})}
\newcommand{\sequence}[2]{\ensuremath{(#2)_{#1 \in \naturals}}}
\newcommand{\highlight}[1]{\textcolor{blue}{#1}}
\newcommand{\norm}[2]{\|#2\|_{#1}}
\newcommand{\abs}[1]{|#1|}
\newcommand{\contoperators}[2]{\ensuremath{\mathcal{L}\left(#1, #2\right)}}
\newcommand{\dualspace}[1]{\contoperators{#1}{\thefield}}
\newcommand{\closedunitball}[1]{\ensuremath{B(0, \mathord{\le} 1, #1)}}
\newcommand{\thefield}{\ensuremath{\mathbb{F}}}
\newcommand{\chain}{\ensuremath{\circ}}
\newcommand{\explainstep}[1]{&& \textcolor{darkgray}{\text{(#1)}}}
\newcommand{\logicaland}{\ensuremath{\wedge}}
\newcommand{\inv}[1]{\ensuremath{#1^{-1}}}


% ===== COMMUNICATION ======
\newcommand{\note}[1]{\textcolor{red}{\text{#1}}}
\newcommand{\comment}[1]{\textcolor{darkgray}{\text{#1}}}

% ===== PROOF DIRECTIONS =====
\newcommand{\leftrightproof}{\glqq\ensuremath{\Rightarrow}\grqq:\quad}
\newcommand{\rightleftproof}{\glqq\ensuremath{\Leftarrow}\grqq:\quad}
\newcommand{\leftrightinclusion}{\glqq\ensuremath{\subset}\grqq:\quad}
\newcommand{\rightleftinclusion}{\glqq\ensuremath{\supset}\grqq:\quad}

% ===== BRACKETS =====
\newcommand{\parens}[1]{\left(#1\right)}
\newcommand{\brackets}[1]{\left[#1\right]}
\newcommand{\braces}[1]{\left\{#1\right\}}

% ===== SETS ======
\newcommand{\reals}{\ensuremath{\mathbb{R}}}
\newcommand{\complexes}{\ensuremath{\mathbb{C}}}
\newcommand{\naturals}{\ensuremath{\mathbb{N}}}

% ===== SPECIAL OPERATORS =====
\DeclareMathOperator{\rk}{rk}
\let\ker\relax
\DeclareMathOperator{\ker}{Ker}
\DeclareMathOperator{\im}{Im}
\DeclareMathOperator{\id}{Id}

% ===== ENUMERATIONS =====
\newlist{romanenum}{enumerate}{1}
\setlist[romanenum]{label=\roman*):, leftmargin=*, labelsep=0.5em, align=left}

\newlist{romanprooflist}{enumerate}{1}
\setlist[romanprooflist]{label="\roman*)":, leftmargin=*, labelsep=0.5em, align=left}


\newenvironment{proofidea}
{\begin{proof}[Proof Idea]}
{\end{proof}}



% ===== THEOREM STYLING =====
\theoremstyle{plain}
\newtheorem{theorem}{Theorem}
\newtheorem{revision}{Revision}
\newtheorem{corollary}{Corollary}
\theoremstyle{definition}
\newtheorem{definition}{Definition}
\newtheorem{example}{Example}
\newtheorem{remark}{Remark}


% ===== MISC =====
\newcommand{\refertolecture}{\text{Please refer to the functional analysis lecture notes from SS/2025.}}


% ===== DOCUMENT ====
\begin{document}
\maketitle

\begin{abstract}
    This proseminar paper provides a foundational overview of dual-/adjoint operators in a general Banach space setting.
\end{abstract}

\tableofcontents

\section{Introduction}
\subsection{Motivation}
The goal of dual- or adjoint operators is to generalize the notion of an adjoint matrix
(often denoted as $A^T$ over \reals\ or $A^H$ over \complexes)
to general operators on banach spaces.

\noindent We will cover and generalize the following topics:
\begin{enumerate}
    \item The operator $\cdot^H$ is linear and isometric wrt. the spectral norm.
    \item The fundamental theorem of linear algebra (Gilbert Strang) or rank-nullity theorem: \\
        $(\im A)^\perp = \ker A^H$.
    \item The lagrange duality from nonlinear optimisation theory.
    \item TODO Add more
\end{enumerate}


\subsection{General revision}

\begin{definition}
    Remember the following concepts:
    \begin{romanenum}
        \item
            Let $\normedspace{X}$ and $\normedspace{Y}$ be normed vector spaces on \reals\ or \complexes. \\
            Let $T: X \rightarrow Y$ be linear. \\
            $T$ is a \highlight{linear operator}.
            $T$ is \highlight{bounded}, if $\exists C > 0, \forall x \in X: \norm{Y}{Tx} \le C \norm{X}{x}$.
        
        \item
            The (topological) dual space is defined as $X' \defeq \contoperators{X}{\mathbb{R}}$.
    \end{romanenum}
\end{definition}

\begin{revision} \label{revision:general-revision}
    The following statements are foundational for this topic:
    \begin{romanenum}
        \item The topological dual space \contoperators{X}{Y} is a Banach space.
        \item For linear operators, continuous and bounded are equivalent.
        \item A linear, bounded operator between $X$ and $Y$ and one between $Y$ and $Z$ can be chained to produce a linear, bounded operator between $X$ and $Z$.
        \item In a chain of linear, bounded operators with at least one compact operator the resulting operator is compact.
    \end{romanenum}
\end{revision}
\begin{proof}
    \refertolecture
\end{proof}

\section{The adjoint operator}
\subsection{The basic definitions and conventions}

\begin{remark}
    The terms \highlight{adjoint} and \highlight{dual} are often used interchangeably.
    We will standardize on \highlight{adjoint}, to avoid unnecessary confusion.
    We will abstract the field as $\thefield \in \{\reals, \complexes\}$.
\end{remark}

\begin{definition}
    Let $X, Y$ be metric spaces and $T \in \contoperators{X}{Y}$. \\
	Then $T': Y' \rightarrow X', y' \mapsto y' \chain T$ is called the adjoint operator.
    From now on, we will implicitly refer to the metric spaces $X, Y$ and the topological dual spaces $X', Y'$ when talking about the dual operator $T'$.
\end{definition}

\begin{remark}
    In comparison to the operators we have previously worked with, the adjoint operator takes and outputs linear, bounded operators.
	It's one more level of abstraction removed from \thefield. \\
	For $y' \in Y' = \dualspace{Y}$, the adjoint operator evaluates to $T' y' \in \dualspace{X} = X'$.
\end{remark}

\begin{example}
    % TODO: Add simple integral operator example.
\end{example}


\subsection{The basic properties}

\noindent The proofs will use the Hahn-Banach corollaries a couple of times.
So first, we need to recall Hahn-Banach related theorems:

\begin{revision}[Hahn-Banach]\label{theorem:hahn-banach-op-norm}
    Let $(X, \norm{}{\cdot})$ be a normed space and $0 \neq x \in X$. \\
    Then we have $\exists f \in X': \norm{}{f} = 1 \logicaland \abs{f(x)} = 1$.
\end{revision}
\begin{proof}
    \refertolecture
\end{proof}

\begin{revision}
    TODO (Explain): \\
    Let $(X, \norm{}{\cdot})$ be a normed vector space and $x \in X$. \\
    Then we have $\norm{X}{x} = \sup_{f \in X', \norm{}{f} \le 1}\abs{f(x)}$.
\end{revision}
\begin{proof}
    Case 1 $x = 0$: Since $X'$ contains linear operators,
    $\norm{X}{x} = 0 = \sup_{f \in X', \norm{}{f} \le 1}\abs{f(0)}$. \\
    Case 2 $x \neq 0$: We have 
    \begin{align*} 
        \sup_{f \in X', \norm{}{f} \le 1}\abs{f(x)}
        \le \sup_{f \in X', \norm{}{f} \le 1}{\norm{}{f}\norm{X}{x}}
        \le 1 \norm{X}{x} = \norm{X}{x}
    \end{align*}
    Using Hahn-Banach from revision \ref{theorem:hahn-banach-op-norm} we get
    $\exists f \in X': \norm{}{f} = 1 \logicaland \abs{f(x)} = \norm{X}{x}$. \\
    So we get $\sup_{f \in X', \norm{}{f} \le 1}\abs{f(x)} = \norm{X}{x}$.
\end{proof}

\begin{theorem}
    We can collect some basic properties of the adjoint operator:
    \begin{romanenum}
        \item $T' \in \contoperators{Y'}{X'}$, so $T'$ is linear and bounded. \\
            \comment{This implies $\forall y' \in \contoperators{Y'}{X'}: T'y' \in X'$}.
        \item $T \mapsto T'$ is linear and isometric.
	 \end{romanenum}
\end{theorem}
\begin{proof}
    \begin{romanprooflist}
        \item
            Let $y' \in Y'$.  Plugging it into the adjoint operator, we get
            $T' y' = y' \chain T$ with signature $X \rightarrow Y \rightarrow \thefield$.
            We can now see that $\im T' \subset X'$.

            \noindent We then prove linearity:
            \begin{align*}
                T'(\alpha y'_1 + y'_2) 
                &= (\alpha y'_1 + y'_2) \chain T                \explainstep{apply def. of dual operator} \\
                &= \alpha y'_1 \chain T + y'_2 \chain T         \explainstep{expand expression} \\
                &= \alpha T' y'_1 + \beta T' y'_2               \explainstep{apply def. of dual operator} \\
                &\text{for all } y'_1, y'_2 \in Y' \text{ and } \alpha \in \thefield
            \end{align*}

            We then prove boundedness of the operator norm:
            \begin{align*}
                \norm{X'}{T' y'}
                &= \norm{X'}{y' \chain T}                           \explainstep{apply def. of dual operator} \\
                &\le \norm{Y'}{y'}\norm{\contoperators{X}{Y}}{T}    \explainstep{apply def. of op. norm} \\
                &\defeq C \norm{Y'}{y'}                             \explainstep{def. the constant} \\
                &\text{for all } y' \in Y'
            \end{align*}

        \item
            We first prove linearity:
            \begin{align*}
                (\alpha T_1 + T_2)'(y')
                &= y'\left(\alpha T_1 + T_2\right)              \explainstep{apply def. of dual operator} \\
                &= \alpha y'(T_1) + y'(T_2)                     \explainstep{y' is linear} \\
                &\text{for all } T_1, T_2 \in \contoperators{X}{Y} 
                \text{ and } y' \in Y' 
                \text{ and } \alpha \in \thefield
            \end{align*}

            We then prove isometry:
            \begin{align*}
                \norm{}{T}
                &= \sup_{\norm{X}{x} \le 1}\norm{Y}{Tx}                                 \explainstep{use supremum char. of op. norm} \\
                &= \sup_{\norm{X}{x} \le 1}\sup_{\norm{}{y'} \le 1}{\abs{y'(Tx)}}       \explainstep{apply theorem of Hahn-Banach\ \ref{theorem:hahn-banach-op-norm}} \\
                &= \sup_{\norm{}{y'} \le 1}\sup_{\norm{X}{x} \le 1}{\abs{y'(Tx)}}       \explainstep{supremum order can be switched} \\
                &= \sup_{\norm{}{y'} \le 1}{\norm{}{T' y'}}                             \explainstep{apply def. of adjoint operator and op. norm} \\
                &= \norm{}{T'}                                                          \explainstep{apply def. of op. norm}
            \end{align*}

    \end{romanprooflist}
\end{proof}

\begin{example}
    % TODO
\end{example}

\newcommand{\lp}{\ensuremath{l_p}\xspace}
\newcommand{\pconj}{\ensuremath{p^*}\xspace}
\newcommand{\lpconj}{\ensuremath{l_\pconj}\xspace}
\begin{example}
    Let $p \in (1, \infty)$ with $p \neq 2$. This makes \lp a Banach space according to the lecture,
    but not a Hilbert space as the parallelogram rule is not satisfied.
    We know that the dual space of \lp is isometrically isomorph to \lpconj
    where \pconj is the \highlight{Hölder conjugate} with $\frac{1}{p} + \frac{1}{p^*} = 1$. \\
    As a reminder, the general idea of the proof of $\lp' \isomorphto \lpconj$ goes as follows:
    \begin{enumerate}
        \item Define the isometric isomorphism as $T: \lpconj \rightarrow \lp', s \mapsto (x \mapsto \sum_{k \in \naturals}{x_k s_k})$.
        \item Verify $(Ts)x$ converges as $\norm{}{(Ts)x} \le \norm{p}{x}\norm{\pconj}{s} < \infty$ and absolute convergence implies convergence.
        \item Verify $T$ is injective using the linearity.
        \item Verify $T$ is surjective and isometric through todo (the annoying part).
    \end{enumerate}

    \noindent Now, let's work through an example! \\
    Consider the \highlight{left shift} operator $T: \lp \rightarrow \lp, \sequence{k}{x_k} \mapsto \sequence{k}{x_{k+1}}$. \\
    It is well-defined since $\sum_{k \in \naturals}\abs{x_{k+1}}^p \le \sum_{k \in \naturals}\abs{x_{k}}^p < \infty$. \\

    \noindent What should the adjoint operator $T'$ intuitively be?

    \noindent The adjoint $T'$ must have the signature $\lp' \isomorphto \lpconj \rightarrow \lp' \isomorphto \lpconj$.

    \noindent Let $y' \in \lp' \isomorphto \lpconj$.
    Then we can write $y': \lp \rightarrow \thefield, x \mapsto \sum_{k \in \naturals}{x_k s_k}$ with $s \in \lpconj$. \\
    Now for $x \in \lp$ we have
    \begin{align*}
        (T' y')(x)
        &= (y' \chain T)(x) = y'(Tx)                \explainstep{apply def. of $T'$} \\
        &= y'(\sequence{k}{x_{k+1}})                \explainstep{apply def. of $T$} \\
        &= \sum_{k \in \naturals}{x_{k+1} s_k}      \explainstep{apply def. of $y'$} \\
        &= \sum_{k \in \naturals}{x_{k} s'_k}       \explainstep{with $s'_1 = 0$ and $s'_k = s_{k-1}$ for $k > 1$}
    \end{align*}

    \noindent This tells us that the adjoint operator $T'$ acts as a \highlight{right shift} (up to isomorphism):
    \begin{align*}
        T': \lpconj \rightarrow \lpconj, \sequence{k}{s_k} \mapsto \sequence{k}{0, s_1, s_2, \cdots}
    \end{align*}
\end{example}

\begin{revision}
    Let $X, Y, Z$ be normed metric spaces. \\
    Then the adjoint operator reverses composition: \\
    $\forall T \in \contoperators{X}{Y}, \forall S \in \contoperators{Y}{Z}: (S \chain T)' = T' \chain S'$.
\end{revision}
\begin{proof}
    Let $T \in \contoperators{X}{Y}$ and $S \in \contoperators{Y}{Z}$. \\
    We know $ST = S \chain T$ is still a linear, bounded operator from $X$ to $Z$.
    So $(ST)'$ is well-defined. \\
    Let $z' \in Z' = \contoperators{Z}{\thefield}$ and set $y' = z' \chain S \in \dualspace{Y}$. \\
    We can now evaluate the expression on $z' \in Z'$:
    \begin{align*}
        (ST)'(z') &= z' \chain (ST)     \explainstep{apply def. of dual operator} \\
        &= (z' \chain S) \chain T       \explainstep{write out chain explicitly} \\
        &= y' \chain T                  \explainstep{subst. y'} \\
        &= T' y'                        \explainstep{apply def. of dual operator} \\
        &= T' (z' \chain S)             \explainstep{subst. y'} \\
        &= T' S' z'                     \explainstep{apply def. of dual operator}
    \end{align*}
    So in total, $(ST)' = T' S'$.
\end{proof}

\begin{example}
    % TODO: Add the example with i_X and i_Y. (Lemma III.4.3)
    % TODO: Or revisit the lp example again?
\end{example}


\subsection{The dual space of the dual space}

\noindent For starters, we need to recall some concepts from the lecture.

\begin{definition}
    Let $X$ be a normed vector space. 
    \begin{romanenum}
        \item $X''$ is called the bidual space.
        \item Let $J_X: X \rightarrow X'', p \mapsto (x' \mapsto x'(p))$.
            $J_X$ is called the \highlight{canonical embedding} from $X$ into $X''$.
    \end{romanenum} 
    So $J_X$ returns a function that evaluates dual space elements at $p \in X$.
\end{definition}

\noindent Figure [todo] illustrates why the bidual space in conjunction with the adjoint operator is interesting.
TODO

\newcommand{\canonicalembinv}[1]{\ensuremath{\inv{J_#1|_{J_#1(#1)}}}}
\begin{theorem}\label{theorem:bidual-adjunct-embedding}
    Let $(X, \norm{X}{\cdot})$ and $(Y, \norm{Y}{\cdot})$ be normed vector spaces.
    Let $T \in \contoperators{X}{Y}$ be a bounded linear operator.
    Then we have: $J_Y \chain T = T'' \chain J_X$. \\
    \comment{Or equivalently: $T|_{J_X(X)} = J_Y \chain T'' \chain \canonicalembinv{X}$.} \\
    \comment{Or equivalently: $T'' = \canonicalembinv{Y} \chain T \chain J_X$.}
\end{theorem}
\begin{proof}
    Before the proof, we can avoid confusion by typing out what $T'$ and $T''$ evaluate to:
    \begin{itemize}
        \item $T': Y' \rightarrow X', y' \mapsto y' \chain T$ is the adjoint operator.
        \item $T'': X'' \rightarrow Y'', x'' \mapsto x'' \chain T'$ is the biadjoint operator.
    \end{itemize}

    \noindent Now first of all, we need to check that the signatures of both sides of the equation match:
    \begin{itemize}
        \item $J_Y: Y \rightarrow Y''$   and $T: X \rightarrow Y$     means $(J_Y \chain T): X \rightarrow Y''$.
        \item $T'': X'' \rightarrow Y''$ and $J_X: X \rightarrow X''$ means $(T'' \chain J_X): X \rightarrow Y''$.
    \end{itemize}

    \noindent Finally, for $p \in X$ and $y' \in Y'$ we have
    \begin{align*}
        \brackets{(J_Y \chain T)(p)}(y')
        &= J_Y(Tp)(y') = y'(Tp)                 \explainstep{subst. $p$ in and apply def. of $J_Y$} \\
        &= (y'T)(p)                             \explainstep{use associativity} \\
        &= (T'y')(p)                            \explainstep{apply def. of $T'$} \\
        &= (x' \mapsto x'(p))(T'y')             \explainstep{pull out subst. function} \\
        &= J_X(p)(T'y')                         \explainstep{recognize this is just $J_X$} \\
        &= T''(J_X(p))(y')                      \explainstep{apply def. use $T''$} \\
        &= \brackets{(T'' \chain J_X)(p)}(y')   \explainstep{use \chain\ notation}
    \end{align*}
\end{proof}

\noindent Secondly, we can answer when a continuous operator between $Y'$ and $X'$ is an adjoint operator.
We need to revise an important corollary from the lecture first.

\begin{revision}\label{theorem:embedding-isomorphism}
    Let $(X, \norm{}{\cdot})$ be a normed vector space. Then we have
    \begin{romanenum}
        \item The canonical embedding $J_X$ is an isometric injective function.
        \item $J: X \rightarrow J_X(X), x \mapsto J_X(x)$ is an isometric isomorphism.
        \item \canonicalembinv{Y} is a bounded, linear operator.
    \end{romanenum}
\end{revision}
\begin{proofidea}
    \begin{romanprooflist}
        \item \refertolecture
        \item Follows by definition of $J$ and from ''i)''.
        \item
            Since $J: X \rightarrow J_X(X), x \mapsto J_X(x)$ is an isometric isomorphism, we have
            \begin{itemize}
                \item \canonicalembinv{Y} inherits linearity from $J$.
                \item $\norm{}{J} = 1$ and therefore $\norm{}{\canonicalembinv{Y}} = 1$.
                    This means \canonicalembinv{Y} is bounded.
                    % TODO: Explain this better!
            \end{itemize}

    \end{romanprooflist}
\end{proofidea}

\begin{theorem}
    Let $S \in \contoperators{Y'}{X'}$ be a continuous, linear operator. \\
    Then we have $\exists T \in \contoperators{X}{Y}: T' = S \iff S'(J_X(X)) \subset J_Y(Y)$.
\end{theorem}
\begin{proof}
    \leftrightproof We have 
    \begin{align*}
        S'(J_X(X))
        &= T''(J_X(X))      \explainstep{substitute in $S = T'$} \\
        &= J_Y(T(X))        \explainstep{use theorem\ \ref{theorem:bidual-adjunct-embedding}} \\
        &\subset J_Y(Y)     \explainstep{use $T(X) \subset Y$}
    \end{align*}

    \noindent \rightleftproof 
    We have $S'(J_X(X)) \subset J_Y(Y)$ and revision\ \ref{theorem:embedding-isomorphism} gives us $J_Y(Y) \isomorphto Y$. \\
    So for $x \in X$ and $y''_x = S'(J_X(x))$ there is a (unique) $y_x \in Y$ with $y''_x = J_Y(y_x)$. \\
    Define $T: X \rightarrow Y, x \mapsto y_x$. We know $T$ exists (and is unique) due to the previous argument.

    \noindent We know $T$ is linear and continuous, as $T = y_\cdot = J_Y^{-1} \chain S' \chain J_X$ 
    and all elements in the chain are bounded, linear operators.

    \noindent Let $y' \in Y'$ and $x \in X$. Lastly, we need to prove that $S = T'$:
    \begin{align*}
        (Sy')(x)
        &= J_X(x)(S y')                                     \explainstep{express using $J_X$} \\
        &= (S' J_X(x))(y') = (S' \chain J_X)(x)(y')         \explainstep{apply def. of ajoint op.} \\
        &= J_Y((\inv{J_Y} \chain S' \chain J_X)(x))(y')     \explainstep{use $J_Y \chain \inv{J_Y} = \id$} \\
        &= y'((\inv{J_Y} \chain S' \chain J_X)(x))          \explainstep{evaluate $J_Y$ at the vector} \\
        &= y'(Tx) = (y' \chain T)(x)                        \explainstep{subst. in $T = J_Y^{-1} \chain S' \chain J_X$} \\
        &= (T' y')(x)                                       \explainstep{apply def. of adjoint op.}
    \end{align*}

\end{proof}

\begin{example}\label{example:operator-is-not-adjoint}
    We have already seen an adjunct operator.
    Using the last theorem, we can find an example for an operator 
    $S \in \contoperators{Y'}{X'}$ that is not an adjoint operator \\
    (i.e.\ with $\nexists T \in \contoperators{X}{Y}: T' = S$).

    \noindent TODO.
    % TODO: A not-adjunct operator :) we can use later!
\end{example}

\begin{theorem}
    Lastly, we get one more property of the adjoint operator:
    $T \mapsto T'$ is not always surjective. \\
    \note{This is not the case with $\cdot^H$ e.g.\ between $\reals^n \rightarrow \reals^n$.}
\end{theorem}
\begin{proof}
    The operator and the adjoint operator have the following signatures:
    \begin{itemize}
        \item $T: X \rightarrow Y$
        \item $T': Y' = \dualspace{Y} \rightarrow X' = \dualspace{X}$
    \end{itemize}
    So the ``not always'' refers to a particular choice of $X$ and $Y$ we need to find.
    Fortunately in example \ref{example:operator-is-not-adjoint} we already found a counterexample!
\end{proof}


% \subsection{In Hilbert Spaces}
% Risz-Frechet for Comparison between Hilbetraum-Adjoint and Banachraum-Adjoint.

% \subsection{Examples for Adjoint Operators on Banach Spaces}

% TODO: Examples for dual operators in l_p, L_p and some identity into the dual-dual space.
% TODO: Make sure to find out which is which.

% TODO: Maybe you can add an example from optimisation?

\subsection{Compact (adjoint) operators}

\noindent For starters, we need to recall and revise some concepts from the lecture.

\begin{definition} % NOTE: Revision.
    The following definitions are revisions from the lecture: \\
    Let $X, Y$ be normed metric spaces.
    \begin{romanenum}
        \item $X$ is relatively compact, if $\forall (x_n)_{n \in \naturals} \subset X \text{ bounded}: (T x_n)_{n \in \naturals} \text{ has a converging subsequence}$.
        \item $T \in \naturals$ is compact, if $T B(0, \le 1, X) \text{ is relatively compact}$.
        \item The rank of T is $\rk T = \dim{T(X)}$.
    \end{romanenum}
\end{definition}

% Explanation of the significance of these definitions:
% TODO: Add examples to this section. The motivation is still not clear enough.
\noindent As a reminder, in metric spaces a subset is compact iff.\ all sequences contain a converging subsequence in that set.
So relatively compact is a relaxed version of compactness: it disposes of the closed requirement.
An important property of the closed unit ball in $\reals^n$ or $\complexes$ is that it is compact.
Therefore, a linear operator between $\thefield^n \rightarrow \thefield^m$ always maps the closed unit ball to a compact set.
But the domain might not be a finite-dimensional normed space and the unit ball of the domain might not be compact either.

Even when we only know that a bounded operator has finite rank, we can see that it is always compact:
In finite dimensions, all norms are equivalent. So $(X, \norm{}{\cdot})$ and $(X, \norm{1}{\cdot})$ have the same open sets.
In finite dimensions, linear algebra tells us that $(X, \norm{1}{\cdot})$ and $(\reals^{\dim{X}}, \norm{1}{\cdot})$ have the same open sets.
So all finite-dimensional normed spaces are topologically equivalent to some space $\reals^n$.
In $\reals^n$, the theorem of Heine-Borel tells us that all bounded subsets are relatively-compact.
Since relatively-compact is a topological property, the theorem transfers to $(X, \norm{}{\cdot})$.
Finally, a finite rank causes the bounded image of the operator to be finite-dimensional, which then causes it to be relatively-compact.
This is a very strong statement, considering that the domain unit ball might not be compact at all.

This finally breaks down when the rank is infinite.
So a compact operator simply ensures that we still can enjoy this finite-dimensional behavior between general banach spaces.

\begin{definition}
    The following definitions are critical for the theorem of \highlight{Arzelà-Ascoli}! \\
    Let $X, Y$ be metric spaces and $M \subset \braces{f: X \rightarrow Y}$.
    \begin{romanenum}
        \item $M$ is uniformly equicontinuous, if \\
            $\forall \epsilon > 0, \exists \delta > 0, \forall T \in M, \forall x, y \in X: d_X(x, y) < \delta \implies d_Y(Tx, Ty) < \epsilon$.
        \item $M$ is pointwise bounded, if $\forall x \in X: \braces{f(x) \mid f \in M} \text{ is bounded}$.
    \end{romanenum}
\end{definition}

\noindent The theorem of \highlight{Arzelà-Ascoli} from the lecture gives us a characterisation of relative compactness 
for the set of continuous functions on a compact metric space using the two previous concepts.
This is interesting, because uniform equicontinouity and pointwise boundedness are (relatively) simple properties 
of the functions \highlight{evaluations} rather than the functions themselves. They can be easily verified to be true.
Note that $C(D) = \braces{f: D \rightarrow \thefield \text{ continuous}}$.

\begin{revision}[Arzelà-Ascoli]
    Let $D$ be a compact metric space and $M \subset C(D)$ with the supremum norm.
    Then we have $M$ is uniformly equicontinuous and pointwise bounded implies $M$ is relatively compact.
\end{revision}
\begin{proofidea}
    \begin{enumerate}
        \item $D$ is seperable, i.e. $\exists D_0 \subset D: \overline{D_0} = D$. Simply set $D_0 = \bigcup_{n \in \naturals}{D_n}$ where $D_n$ is a finite $\frac{1}{n}$-covers of $D$.
        \item For all $x \in M$ we can use the pointwise boundedness to invoke Bolzano-Weierstraß to find converging subsequences $(f_{n(x, k)}(x))_{k \in \naturals} \subset \thefield$.
        \item Specifically, we have $\forall x \in D_0 \cap M \eqdef M_0: (f_{n(x, k)}(x))_{k \in \naturals} \text{ converges}$.
        \item Using the commonly used diagonal argument from the lecture, we can find a unified subsequence with no dependence on $x$:
            $\forall x \in M_0: (f_{n(k)}(x))_{k \in \naturals} \text{ converges}$.
        \item We already have $f_{n(k)}|_{M_0} \convergesto: f|_{M_0}$.  And it seems sensible to assume that $(f_{n(k)})_{k \in \naturals}$ is a convergent subsequence.
            But how can you extend this to the entire set $M$?
        \item For each $x \in M$, we can choose an arbitrarily close $x_0 \in M_0$. Using two triangle inequalities, uniform equicontinouity allows us to extend the result to $M$. \\
            As $C(D)$ is complete, the Cauchy sequence converges.
    \end{enumerate}
\end{proofidea}

\noindent Using the revision, we can now prove the theorem of Schauder and then trace the arguments through both proofs to get a better understanding of the ideas.

\begin{theorem}[Schauder]
    Let $X, Y$ be Banach spaces and $T: X \rightarrow Y$ be a bounded, linear operator.
    Then we have $T$ is compact iff.\ $T'$ is compact.
\end{theorem}
\begin{proof}
    \leftrightproof
    Let $(y'_n)_{n \in \naturals} \subset Y' = \dualspace{Y} \subset C(Y)$ be bounded. \\
    Our goal is to show that there is a convergent subsequence in $(T' y'_n)_{n \in \naturals}$
    with respect to $(X', \norm{}{\cdot})$ where $\norm{}{\cdot}$ is the operator norm.

    \noindent Let $K = \closedunitball{X}$ and for all $n \in \naturals$ set $f_n = (y'_n \chain T)|_{B} \in \dualspace{X}$.
    Then we have
    \begin{align*}
        \norm{\infty}{f - f_m} % TODO: This part of the proof is very questionable. I think this is wrong!
        &= \sup_{\norm{X}{x} = 1}{((y' \chain T) - (y'_m \chain T))x}       \explainstep{apply def. of $f_n$ and $f_m$} \\
        &= \norm{}{y' \chain T - y'_m \chain T}                             \explainstep{use supremum char. of operator norm}
    \end{align*}
    This tells us that convergence in the operator norm only cares about the behavior on the closed unit ball!
    Let $D = \overline{T \closedunitball{X}}$.
    \noindent We have
    \begin{enumerate}
        \item $D$ is an (inherited) metric space.
        \item $D$ is closed by definition.
        \item $T(\closedunitball{X})$ is a bounded set under a bounded operator, so $D$ is bounded.
    \end{enumerate}
    Since $D$ is in a metric space, it must be compact. \\

    \noindent We can now pack our sequence into $M \defeq \braces{f_n \mid n \in \naturals}$ and examine it for
    \begin{enumerate}
        \item Pointwise boundedness: For $x \in \closedunitball{X}$ and $n \in \naturals$
            \begin{align*}
                \abs{f_n(x)}
                &= \abs{y'_n(Tx)} \defeq \abs{y'_n(d)}      \explainstep{apply def. of $f_n$ and define $d$} \\
                &\le C_1 \norm{Y}{d} \le C_1 C_2            \explainstep{apply op. norm ineq. and D bounded means $\norm{Y}{d} \le: C_2$}
            \end{align*}

        \item Uniform equicontinuity: For $n \in \naturals, \epsilon > 0, \delta = \frac{\epsilon}{C_1 C_2}, \forall x, y \in K, \norm{X}{x - y} < \delta$
            \begin{align*}
                \norm{Y}{f_n(x) - f_n(y)}
                &\le \norm{Y}{y'_n \chain Tx - y'_n \chain Ty}                  \explainstep{apply def. of $f_n$} \\
                &\le \norm{}{y'_n} \norm{Y}{T(x - y)}                           \explainstep{factor out and apply op. norm ineq.} \\
                &\le \norm{}{y'_n} \norm{}{T} \norm{X}{x - y}                   \explainstep{apply op. norm ineq.} \\
                &\le C_1 C_2 \norm{X}{x - y} < C_1 C_2 \delta = \epsilon        \explainstep{apply bounded and substitute in $\delta$}
            \end{align*}
    \end{enumerate}

    \noindent The theorem of Arzelà-Ascoli now tells us that $M$ is relatively compact.
    So every sequence in $M$ has a convergent subsequence.
    In particular for the convergent subsequence \sequence{k}{f_{n(k)}} we have
    \begin{align*}
        \sequence{k}{f_{n(k)}}
        &= \sequence{k}{y'_n \chain T}      \explainstep{apply def. of $f_{n(k)}$} \\
        &= \sequence{k}{T' y'_{n(k)}}       \explainstep{apply def. of adjoint operator}
    \end{align*}

    \noindent So finally, \sequence{k}{T' y'_{n(k)}} is a convergent subsequence of \sequence{k}{T' y'_k}. \\

    \noindent \rightleftproof
    The other proof direction tells us that $T \text{ compact } \implies T' \text{ compact}$. \\
    So in extension this also yields $T' \text{ compact } \implies T'' \text{ compact}$.

    \noindent Theorem \ref{theorem:bidual-adjunct-embedding} tells us that $T = \inv{J_Y} \chain T'' \chain J_X$.

    \noindent And revision\ \ref{revision:general-revision} tells us that $T$ is also compact,
    as it is a composition of at least one compact operator- in this case $T''$.

\end{proof}

\begin{example}
    % Example with C[0, 1] and a compact integral operator on that space.
\end{example}

% IDK if this fits in at all????
% \begin{example}
%     % General example with compact integral operator (Script Example 3.35)
%     % Specify to bounded fourier transform.
%     % Somehow connect to dual topological space??? Idk if this is good. The punchline is maybe not ideal.
% \end{example}

\subsection{The rank-nullity theorem for operators}
\newcommand{\anihu}{\ensuremath{U^\perp}}
\newcommand{\anihv}{\ensuremath{V_\perp}}

\begin{definition}
    Let $(X, \norm{}{\cdot})$ be a normed vector space, $U \subset X$ and $V \subset X'$.
    Then define
    \begin{romanenum}
        \item $\anihu = \braces{x' \in X' \mid \forall x \in U: x'(x) = 0}$ as the annihilator of $U$ in $X'$.
        \item $\anihv = \braces{x \in X \mid \forall x' \in V: x'(x) = 0}$ as the annihilator of $V$ in $X$.
    \end{romanenum}
\end{definition}

\noindent To prove properties about these sets, we need a corollary
of the theorem of Hahn-Banach.

\begin{revision}
    TODO: Hahn-Banach corollary.
\end{revision}
\begin{proofidea}
    TODO
\end{proofidea}

\begin{theorem}
    Let $(X, \norm{}{\cdot})$ be a normed vector space, $U \subset X$ and $V \subset X'$.
    Then we have
    \begin{romanenum}
        \item $\anihu$ and $\anihv$ are not empty.
        \item $\anihu \subset X'$ and $\anihv \subset X$ are both respectively closed linear subspaces.
        \item Let $U$ be closed. Then we have $(X / U)' \isomorphto \anihu$ (they are isomorphic).
        \item Let $U$ be closed. Then we have $U' \isomorphto X' / \anihu$ (they are isomorphic).
    \end{romanenum}
\end{theorem}
\begin{proof}
    \begin{romanprooflist}
        \item
            

        \item 
            We will focus on $\anihu \subset X'$. The other statement works analogously. \\
            Let $x'_1, x'_2 \in X'$ and $\lambda \in \thefield$.
            We first prove $\emptyset \neq \anihu \subset X'$ is a linear subspace:
            \begin{align*}
                (x'_1 + \lambda x'_2)(x)
                &= x'_1(x) + \lambda x'_2(x)    \explainstep{substitute $x$ in} \\
                &= 0 + \lambda 0 = 0            \explainstep{use $x'_1, x'_2 \in X'$}
            \end{align*}

            \noindent Let $\sequence{k}{x'_k} \subset \anihu$ be a converging sequence. \\
            We now prove that $x'$ converges in $\anihu$:
            \begin{align*}
                todo
            \end{align*}

        \item 

        \item 

    \end{romanprooflist}
\end{proof}

\noindent In summary, TODO.

\newcommand{\kerdualperp}{\ensuremath{(\ker T')_\perp}}
\begin{theorem}
    Let $T \in \contoperators{X}{Y}$ be a bounded, linear operator.
    Then we have $\overline{\im T} = \kerdualperp$.
\end{theorem}
\begin{proof}
    \leftrightinclusion
    Let $Tx \in \im T$ with $x \in X$ and $y' \in \ker T'$. \\
    We first prove $Tx \in \kerdualperp$:
    \begin{align*}
        y'Tx &= (y' \chain T)(x)     \explainstep{use \chain\ notation} \\
        &= (T' y')(x)               \explainstep{apply dual op. definition} \\
        &= 0(x) = 0                 \explainstep{apply $y' \in \ker T'$}
    \end{align*}
    Since this holds for all choices of $Tx$ we get $\im T \subset \kerdualperp$. \\
    Since \kerdualperp is closed, we also get $\overline{\im T} \subset \kerdualperp$.

    \noindent \rightleftinclusion
    TODO
\end{proof}

\begin{theorem}
    TODO: The stuff with the Id - T operator.
\end{theorem}
\begin{proof}
    TODO
\end{proof}

\begin{corollary}
    Let $T \in \contoperators{X}{Y}$ be a linear, continuous operator with $\im T$ closed.
    Then we have $Tx = y$ has a solution iff.\ $T'y' = 0 \implies y'(y) = 0$.
\end{corollary}
\begin{proof}
\end{proof}

\begin{example}
\end{example}

% TODO: Theorem about solutions to Tx=y
% TODO: Example as to how it relates to Lagrange Multiplicators in Optimisation

\nocite{*}
\printbibliography

% TODO: Add reminders everywhere. E.g. remind the reader of the definition of topological dual spaces when appropriate. We need to assume the reader is new to the subject.

% TODO: Linear subspace proofs require showing the set is not empty!

% TODO: Download final bibliography from zotero.

% TODO: Type names large.

% TODO: Flag any occurences of behaviour that is different from the finite dimensional easy cases.

% TODO: Make sure all theorems, corollaries and revisions are linked properly!

\end{document}
