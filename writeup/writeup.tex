\documentclass{article}

\usepackage[english]{babel}
\usepackage[a4paper,top=2cm,bottom=2cm,left=3cm,right=3cm,marginparwidth=1.75cm]{geometry}
\usepackage{amsmath, amsfonts, graphicx, enumitem}
\usepackage[colorlinks=true, allcolors=blue]{hyperref}
\usepackage{dsfont, mathtools, amsthm, diagbox, caption, subcaption}
\usepackage[
    backend=bibtex,
    style=alphabetic,
]{biblatex}
\addbibresource{../bibliography.bib} %Imports bibliography file

\title{Introduction to Entropy}
\author{Erik Stern}

\newcommand{\defeq}{\vcentcolon=}
\newcommand{\eqdef}{=\vcentcolon}
\newcommand{\paren}[1]{\left(#1\right)}
\newcommand{\normedspace}[1]{(#1, \|\cdot\|_{#1})}
\newcommand{\highlight}[1]{\textcolor{blue}{\textit{#1}}}
\newcommand{\norm}[2]{\|#2\|_{#1}}
\newcommand{\topodual}[2]{\ensuremath{\mathcal{L}\left(#1, #2\right)}}

\newenvironment{romanenum}
{\begin{enumerate}[label=\roman*)]}
{\end{enumerate}}

\newenvironment{proofidea}
{\begin{proof}[Proof Idea]}
{\end{proof}}

\theoremstyle{plain}
\newtheorem{theorem}{Theorem}
\newtheorem{corollary}{Corollary}

\theoremstyle{definition}
\newtheorem{definition}{Definition}
\newtheorem{example}{Example}
\newtheorem{remark}{Remark}

\begin{document}
\maketitle

\begin{abstract}
    This seminar paper provides a foundational overview of Dual-/Adjunct-Operators in functional analysis.
\end{abstract}

\tableofcontents

\section{Revision}

\begin{definition}
    The following basic definitions:
    \begin{romanenum}
        \item Banach Space
        \item Hilbert Space
    \end{romanenum}
\end{definition}

\begin{definition}
    Let $\normedspace{X}$ and $\normedspace{Y}$ be normed vector spaces on $\mathbb{R}$ or $\mathbb{C}$. \\
    Let $T: X \rightarrow Y$ be linear. \\
    $T$ is a \highlight{linear operator}.
    $T$ is \highlight{bounded}, if $\exists C > 0, \forall x \in X: \norm{Y}{Tx} \le C \norm{X}{x}$.
\end{definition}

\begin{definition}
    \begin{romanenum}
        \item Algebraic dual space: $X' \defeq \mathcal{L}$.
        \item (Topological) Dual space: $X* \defeq \mathcal{L}$.
    \end{romanenum}
\end{definition}

\begin{theorem}
    The topological dual space \topodual{X}{Y} is a Banach space.
\end{theorem}
\begin{proofidea}
    
\end{proofidea}

\begin{theorem}
    For linear operators, continous and bounded are equivalent.
\end{theorem}
\begin{proofidea}
    TOOD
\end{proofidea}

\begin{definition}
    \begin{romanenum}
        \item 
        \item 
    \end{romanenum}
\end{definition}

\begin{theorem}
    \begin{romanenum}
        \item % l_p isomorphic to ...
            Let $p \in [1, \infty)$.
            Then 
        \item % L_p isomorphic to ...
    \end{romanenum}
\end{theorem}
\begin{proofidea}
    TOOD
\end{proofidea}


\begin{theorem}
\end{theorem}
\begin{proof}
    See functional analysis lecture of SS/25.
\end{proof}


% TODO: Sublinear Definition and Theorem of Hahn-Banach


\newpage
\section{The adjunct operator}


\begin{definition}
    Let $X, Y$ be metric spaces and $T \in \topodual{X}{Y}$.
    Let $T': Y' -> X'$.
    Then $T'$ is the adjunct-/dual operator, 
    if $(T'y')(x) = y'(Tx)$.
\end{definition}

% TODO: Examples for dual operators in l_p, L_p and some identity into the dual-dual space.

% TODO: Maybe you can add an example from optimisation?

% TODO: Thereom by Arzela-Ascoli

\begin{theorem}

\end{theorem}

\newpage
\printbibliography

\end{document}
