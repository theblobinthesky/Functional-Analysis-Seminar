\subsection{Compact (adjoint) operators}

\noindent For starters, we need to recall and revise some concepts from the lecture.

\begin{definition} % NOTE: Revision.
    The following definitions are revisions from the lecture: \\
    Let $X, Y$ be normed \thefield\ vector spaces.
    \begin{romanenum}
        \item $X$ is relatively compact, if \[\forall (x_n)_{n \in \naturals} \subset X \text{ bounded}: (x_n)_{n \in \naturals} \text{ has a converging subsequence in } Y\]
        \item $T \in \contoperators{X}{Y}$ is compact, if \[T\parens{\closedunitball{X}} \text{ is relatively compact}\]
        \item The rank of T is $\rk T = \dim{T(X)}$.
    \end{romanenum}
\end{definition}

% Explanation of the significance of these definitions:
% TODO: Add examples to this section. The motivation is still not clear enough.
\noindent As a reminder, in metric spaces a subset is compact if and only if all sequences contain a converging subsequence in that set.
So relatively compact is a relaxed version of compactness: it disposes of the closed requirement.
An important property of the closed unit ball in $\reals^n$ or $\complexes$ is that it is compact.
Therefore, a linear operator between $\thefield^n \rightarrow \thefield^m$ always maps the closed unit ball to a compact set.
But the domain might not be a finite-dimensional normed space and the unit ball of the domain might not be compact either.

Even when we only know that a bounded operator has finite rank, we can see that it is always compact:
In finite dimensions, all norms are equivalent. So $(X, \norm{}{\cdot})$ and $(X, \norm{1}{\cdot})$ have the same open sets.
In finite dimensions, linear algebra tells us that $(X, \norm{1}{\cdot})$ and $(\reals^{\dim{X}}, \norm{1}{\cdot})$ have the same open sets.
So all finite-dimensional normed spaces are topologically equivalent to some space $\reals^n$.
In $\reals^n$, the theorem of Heine-Borel tells us that all bounded subsets are relatively compact.
Since relatively compact is a topological property, the theorem transfers to $(X, \norm{}{\cdot})$.
Finally, a finite rank causes the bounded image of the operator to be finite-dimensional, which then causes it to be relatively compact.
This is a very strong statement, considering that the domain unit ball might not be compact at all.

This finally breaks down when the rank is infinite.
So a compact operator simply ensures that we still can enjoy this finite-dimensional behavior between general Banach spaces.

\begin{definition}
    The following definitions are critical for the theorem of \highlight{Arzelà-Ascoli}: \\
    Let $X, Y$ be metric spaces and $M \subset \braces{f: X \rightarrow Y}$.
    \begin{romanenum}
        \item $M$ is uniformly equicontinuous, if
            \[ \forall \epsilon > 0, \exists \delta > 0, \forall T \in M, \forall x, y \in X: d_X(x, y) < \delta \implies d_Y(Tx, Ty) < \epsilon \]
        \item $M$ is pointwise bounded, if 
            \[ \forall x \in X: \braces{f(x) \mid f \in M} \text{ is bounded} \]
    \end{romanenum}
\end{definition}

\noindent The theorem of \highlight{Arzelà-Ascoli} from the lecture gives us a characterisation of relative compactness 
for the set of continuous functions on a compact metric space using the two previous concepts.
This is interesting, because uniform equicontinouity and pointwise boundedness are (relatively) simple properties 
of the functions \highlight{evaluations} rather than the functions themselves. They can be easily verified to be true.
Note that $C(D) = \braces{f: D \rightarrow \thefield \text{ continuous}}$.

\begin{revision}[Arzelà-Ascoli]
    Let $D$ be a compact metric space and $M \subset C(D)$ with the supremum norm.
    Then we have $M$ is uniformly equicontinuous and pointwise bounded implies $M$ is relatively compact.
\end{revision}
\begin{proofidea}
    \begin{itemize}
        \item $D$ is separable, i.e. $\exists D_0 \subset D: \overline{D_0} = D$. Simply set $D_0 = \bigcup_{n \in \naturals}{D_n}$ where $D_n$ is a finite $\frac{1}{n}$-covers of $D$.
        \item For all $x \in D$ we can use the pointwise boundedness to invoke Bolzano-Weierstraß to find converging subsequences $(f_{n(x, k)}(x))_{k \in \naturals} \subset \thefield$.
        \item Specifically, we have $\forall x \in D_0: (f_{n(x, k)}(x))_{k \in \naturals} \text{ converges}$.
        \item Using the commonly used diagonal argument from the lecture, we can find a unified subsequence with no dependence on $x$:
            $\forall x \in D_0: (f_{n(k)}(x))_{k \in \naturals} \text{ converges}$.
        \item We already have $f_{n(k)}|_{D_0} \convergesto: f|_{D_0}$.  And it seems sensible to assume that $(f_{n(k)})_{k \in \naturals}$ is a convergent subsequence.
            % TODO: \convergesto: is bad notation
            But how can you extend this to the entire set $D$?
        \item For each $x \in D$, we can choose an arbitrarily close $x_0 \in D_0$. Using two triangle inequalities, uniform equicontinouity allows us to extend the result to $D$. \\
            As $C(D)$ is complete, the Cauchy sequence converges.
    \end{itemize}
\end{proofidea}

\noindent Using the revision, we can now prove the theorem of Schauder and then trace the arguments through both proofs to get a better understanding of the ideas.

\begin{theorem}[Schauder]
    Let $X, Y$ be Banach spaces and $T: X \rightarrow Y$ be a bounded, linear operator.
    Then we have $T$ is compact if and only if $T'$ is compact.
\end{theorem}
\begin{proof}
    TODO the proof here is (slightly) still wrong bc of the way the f's are defined!
    But conceptually, should be good!
    \leftrightproof
    Let $(y'_n)_{n \in \naturals} \subset Y' = \dualspace{Y} \subset C(Y)$ be bounded. \\
    Our goal is to show that there is a convergent subsequence in $(T' y'_n)_{n \in \naturals}$
    with respect to $(X', \norm{}{\cdot})$ where $\norm{}{\cdot}$ is the operator norm.

    \noindent Let $K = \closedunitball{X}$ and for all $n \in \naturals$ set $f_n = (y'_n \chain T)|_{B} \in \dualspace{X}$.
    Then we have
    \begin{align*}
        \norm{\infty}{f - f_m} % TODO: This part of the proof is very questionable. I think this is wrong!
        &= \sup_{\norm{X}{x} = 1}{((y' \chain T) - (y'_m \chain T))x}       \explainstep{apply def. of $f_n$ and $f_m$} \\
        &= \norm{}{y' \chain T - y'_m \chain T}                             \explainstep{use supremum char. of operator norm}
    \end{align*}
    This tells us that convergence in the operator norm only cares about the behavior on the closed unit ball!
    Let $D = \overline{T\parens{\closedunitball{X}}}$.
    \noindent We know
    \begin{itemize}
        \item \closedunitball{X} is bounded.
        \item $T$ is a compact operator.
    \end{itemize}
    So $T \closedunitball{X}$ is relatively compact and $D$ is compact.

    \noindent We can now pack our sequence into $M \defeq \braces{f_n \mid n \in \naturals}$ and examine it for
    \begin{itemize}
        \item Pointwise boundedness: For $x \in \closedunitball{X}$ and $n \in \naturals$
            \begin{align*}
                \abs{f_n(x)}
                &= \abs{y'_n(Tx)} \defeq \abs{y'_n(d)}      \explainstep{apply def. of $f_n$ and define $d$} \\
                &\le C_1 \norm{Y}{d} \le C_1 C_2            \explainstep{apply op. norm ineq. and D bounded means $\norm{Y}{d} \le: C_2$}
            \end{align*}

        \item Uniform equicontinuity: For $n \in \naturals, \epsilon > 0, \delta = \frac{\epsilon}{C_1 C_2}, \forall x, y \in K, \norm{X}{x - y} < \delta$
            \begin{align*}
                \norm{Y}{f_n(x) - f_n(y)}
                &\le \norm{Y}{y'_n \chain Tx - y'_n \chain Ty}                  \explainstep{apply def. of $f_n$} \\
                &\le \norm{}{y'_n} \cdot \norm{Y}{T(x - y)}                     \explainstep{factor out and apply op. norm ineq.} \\
                &\le \norm{}{y'_n} \cdot \norm{}{T} \cdot \norm{X}{x - y}       \explainstep{apply op. norm ineq.} \\
                &\le C_1 C_2 \norm{X}{x - y} < C_1 C_2 \delta = \epsilon        \explainstep{apply bounded and substitute in $\delta$}
            \end{align*}
    \end{itemize}

    \noindent The theorem of Arzelà-Ascoli now tells us that $M$ is relatively compact.
    So every sequence in $M$ has a convergent subsequence.
    In particular for the convergent subsequence \sequence{k}{f_{n(k)}} we have
    \begin{align*}
        \sequence{k}{f_{n(k)}}
        &= \sequence{k}{y'_n \chain T}      \explainstep{apply def. of $f_{n(k)}$} \\
        &= \sequence{k}{T' y'_{n(k)}}       \explainstep{apply def. of adjoint operator}
    \end{align*}

    \noindent So finally, \sequence{k}{T' y'_{n(k)}} is a convergent subsequence of \sequence{k}{T' y'_k}. \\

    \noindent \rightleftproof
    The other proof direction tells us that $T \text{ compact } \implies T' \text{ compact}$. \\
    So in extension this also yields $T' \text{ compact } \implies T'' \text{ compact}$.

    \noindent Theorem \ref{theorem:bidual-adjunct-embedding} tells us that $T = \inv{J_Y} \chain T'' \chain J_X$.

    \noindent And revision\ \ref{revision:general-revision} tells us that $T$ is also compact,
    as it is a composition of at least one compact operator - in this case $T''$.

\end{proof}

\begin{example}
    % Example with C[0, 1] and a compact integral operator on that space.
\end{example}

% IDK if this fits in at all????
% \begin{example}
%     % General example with compact integral operator (Script Example 3.35)
%     % Specify to bounded fourier transform.
%     % Somehow connect to dual topological space??? Idk if this is good. The punchline is maybe not ideal.
% \end{example}
