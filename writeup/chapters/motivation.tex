\section{Motivation}
The goal of dual- or adjoint operators is to generalize the notion of an adjoint matrix
(often denoted as $A^T$ over \reals\ or $A^H$ over \complexes)
to operators on normed \reals or \complexes vector spaces:

\noindent We will cover the following topics:
\begin{itemize}
    \item The general version of the operator $\cdot^H$ is linear and isometric wrt. the spectral norm. % TODO: Explain what .^H is.
    \item The general version of the fundamental theorem of linear algebra (Gilbert Strang) or rank-nullity theorem: \\
        $(\im A)^\perp = \ker A^H$.
    \item How the adjunct operation behaves on compact operators.
    \item A characterisation of $\exists x \in X: Tx = y$.
\end{itemize}

