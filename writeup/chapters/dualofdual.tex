\subsection{The dual space of the dual space}

\noindent For starters, we need to recall some concepts from the lecture.

\begin{definition}
    Let $X$ be a normed \thefield\ vector space. 
    \begin{romanenum}
        \item $X''$ is called the bidual space.
        \item Let $J_X: X \rightarrow X'', p \mapsto (x' \mapsto x'(p))$.
            $J_X$ is called the \highlight{canonical embedding} from $X$ into $X''$.
    \end{romanenum} 
    So $J_X$ returns a function that evaluates dual space elements at $p \in X$.
\end{definition}

\noindent Figure [todo] illustrates why the bidual space in conjunction with the adjoint operator is interesting.
TODO

\newcommand{\canonicalembinv}[1]{\ensuremath{\inv{J_#1|_{J_#1(#1)}}}}
\begin{theorem}\label{theorem:bidual-adjunct-embedding}
    Let $(X, \norm{X}{\cdot})$ and $(Y, \norm{Y}{\cdot})$ be normed \thefield\ vector spaces.
    Let $T \in \contoperators{X}{Y}$ be a bounded linear operator.
    Then we have: $J_Y \chain T = T'' \chain J_X$. \\
    \comment{Or equivalently: $T''|_{J_X(X)} = J_Y \chain T \chain \canonicalembinv{X}$.} \\
    \comment{Or equivalently: $T = \canonicalembinv{Y} \chain T'' \chain J_X$.}
\end{theorem}
\begin{proof}
    Before the proof, we can avoid confusion by typing out what $T'$ and $T''$ evaluate to:
    \begin{itemize}
        \item $T': Y' \rightarrow X', y' \mapsto y' \chain T$ is the adjoint operator.
        \item $T'': X'' \rightarrow Y'', x'' \mapsto x'' \chain T'$ is the biadjoint operator.
    \end{itemize}

    \noindent Now first of all, we need to check that the signatures of both sides of the equation match:
    \begin{itemize}
        \item $J_Y: Y \rightarrow Y''$   and $T: X \rightarrow Y$     means $(J_Y \chain T): X \rightarrow Y''$.
        \item $T'': X'' \rightarrow Y''$ and $J_X: X \rightarrow X''$ means $(T'' \chain J_X): X \rightarrow Y''$.
    \end{itemize}

    \noindent Finally, for $p \in X$ and $y' \in Y'$ we have
    \begin{align*}
        \parens{(J_Y \chain T)(p)}(y')
        &= J_Y(Tp)(y') = y'(Tp)                 \explainstep{subst. $p$ in and apply def. of $J_Y$} \\
        &= (y'T)(p)                             \explainstep{use associativity} \\
        &= (T'y')(p)                            \explainstep{apply def. of $T'$} \\
        &= (x' \mapsto x'(p))(T'y')             \explainstep{pull out subst. function} \\
        &= J_X(p)(T'y')                         \explainstep{recognize this is just $J_X$} \\
        &= T''(J_X(p))(y')                      \explainstep{apply def. use $T''$} \\
        &= \parens{(T'' \chain J_X)(p)}(y')   \explainstep{use \chain\ notation}
    \end{align*}
\end{proof}

\noindent Secondly, we can answer when a continuous operator between $Y'$ and $X'$ is an adjoint operator.
We need to revise an important corollary from the lecture first.

\begin{revision}\label{theorem:embedding-isomorphism}
    Let $(X, \norm{}{\cdot})$ be a normed \thefield\ vector space. Then we have
    \begin{romanenum}
        \item The canonical embedding $J_X$ is an isometric injective function.
        \item $J: X \rightarrow J_X(X), x \mapsto J_X(x)$ is an isometric isomorphism.
        \item \canonicalembinv{X} is a bounded, linear operator.
    \end{romanenum}
\end{revision}
\begin{proofidea}
    \begin{romanprooflist}
        \item \refertolecture
        \item Follows by definition of $J$ and from ''i)''.
        \item
            Since $J: X \rightarrow J_X(X), x \mapsto J_X(x)$ is an isometric isomorphism, we have
            \begin{itemize}
                \item \canonicalembinv{X} inherits linearity from $J$.
                \item $\norm{}{J} = 1$ and therefore $\norm{}{\canonicalembinv{X}} = 1$.
                    This means \canonicalembinv{X} is bounded.
                    % TODO: Explain this better!
            \end{itemize}

    \end{romanprooflist}
\end{proofidea}

\begin{theorem}
    Let $S \in \contoperators{Y'}{X'}$ be a continuous, linear operator. \\
    Then we have $\exists T \in \contoperators{X}{Y}: T' = S \iff S'(J_X(X)) \subset J_Y(Y)$.
\end{theorem}
\begin{proof}
    \leftrightproof We have 
    \begin{align*}
        S'(J_X(X))
        &= T''(J_X(X))      \explainstep{substitute in $S = T'$} \\
        &= J_Y(T(X))        \explainstep{use theorem\ \ref{theorem:bidual-adjunct-embedding}} \\
        &\subset J_Y(Y)     \explainstep{use $T(X) \subset Y$}
    \end{align*}

    \noindent \rightleftproof 
    We have $S'(J_X(X)) \subset J_Y(Y)$ and revision\ \ref{theorem:embedding-isomorphism} gives us $J_Y(Y) \isomorphto Y$. \\
    So for $x \in X$ and $y''_x = S'(J_X(x))$ there is a (unique) $y_x \in Y$ with $y''_x = J_Y(y_x)$. \\
    Define $T: X \rightarrow Y, x \mapsto y_x$. We know $T$ exists (and is unique) due to the previous argument.

    \noindent We know $T$ is linear and continuous, as $T = y_\cdot = J_Y^{-1} \chain S' \chain J_X$ 
    and all elements in the chain are bounded, linear operators.
    Let $y' \in Y'$ and $x \in X$. Lastly, we need to prove that $S = T'$:
    \begin{align*}
        (Sy')(x)
        &= J_X(x)(S y')                                             \explainstep{express using $J_X$} \\
        &= (S' J_X(x))(y') = (S' \chain J_X)(x)(y')                 \explainstep{apply def. of ajoint op.} \\
        &= J_Y((\canonicalembinv{Y} \chain S' \chain J_X)(x))(y')   \explainstep{use $J_Y \chain \canonicalembinv{Y} = \id$} \\
        &= y'((\canonicalembinv{Y} \chain S' \chain J_X)(x))        \explainstep{evaluate $J_Y$ at the vector} \\
        &= y'(Tx) = (y' \chain T)(x)                                \explainstep{subst. in $T = J_Y^{-1} \chain S' \chain J_X$} \\
        &= (T' y')(x)                                               \explainstep{apply def. of adjoint op.}
    \end{align*}

\end{proof}

\begin{example}\label{example:operator-is-not-adjoint}
    We have already seen an adjunct operator.
    Using the last theorem, we can find an example for an operator 
    $S \in \contoperators{Y'}{X'}$ that is not an adjoint operator \\
    (i.e.\ with $\nexists T \in \contoperators{X}{Y}: T' = S$).

    \noindent TODO.
    % TODO: A not-adjunct operator :) we can use later!
\end{example}

% TODO: We should probably merge the example with the theorem.
\begin{theorem}
    Lastly, we get one more property of the adjoint:
    $T \mapsto T'$ is not always surjective. \\
    \note{This is not the case with $\cdot^H$ e.g.\ between $\reals^n \rightarrow \reals^n$.}
\end{theorem}
\begin{proof}
    The operator and the adjoint operator have the following signatures:
    \begin{itemize}
        \item $T: X \rightarrow Y$
        \item $T': Y' = \dualspace{Y} \rightarrow X' = \dualspace{X}$
    \end{itemize}
    So the ``not always'' refers to a particular choice of $X$ and $Y$ we need to find.
    Fortunately in example \ref{example:operator-is-not-adjoint} we already found a counterexample!
\end{proof}


% \subsection{Examples for Adjoint Operators on Banach Spaces}

% TODO: Examples for dual operators in l_p, L_p and some identity into the dual-dual space.
% TODO: Make sure to find out which is which.

% TODO: Maybe you can add an example from optimisation?
