\begin{titlepage}
    \centering
    \null
    \vspace{2.0cm}
    \includegraphics[width=0.4\textwidth]{../uni-passau-logo.png} \\
    \vspace{0.5cm}
    {\scshape\LARGE University of Passau \par}
    \vspace{0.5cm}
    {\scshape\Large Faculty of Computer Science and Mathematics \par}
    \vspace{0.2cm}
    {\large Chair of Functional Analysis \par}
    \vspace{1.5cm}
    {\huge\bfseries Dual Operators \par}
    \vspace{0.5cm}
    {\Large Seminar Functional Analysis \par}
    \vspace{1cm}
    {\Large Erik Stern \par}
    \vspace{0.5cm}
    {\Large January 2026 \par}
    \vspace{1.0cm}
    \raggedright
    {\noindent\textbf{Abstract:}
        This seminar paper summarizes the theory of adjoint operators for bounded linear operators between Banach spaces.
        We establish fundamental properties, showing that the adjoint mapping is linear, isometric, and reverses composition.
        The relationship between an operator and its bidual is explored through the canonical embedding, leading to a characterization of when an operator between dual spaces arises as an adjoint.
        The central result is Schauder's theorem, which establishes that compactness is preserved under taking adjoints.
        Finally, we generalize the rank-nullity theorem from linear algebra to the infinite-dimensional setting and illustrate the theory with shift operators on sequence spaces.
    }
    \vfill
\end{titlepage}