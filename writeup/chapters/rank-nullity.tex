\subsection{The rank-nullity theorem for operators}
\newcommand{\anihu}{\ensuremath{U^\perp}}
\newcommand{\anihv}{\ensuremath{V_\perp}}

\begin{definition}
    Let $(X, \norm{}{\cdot})$ be a normed \thefield\ vector space, $U \subset X$ and $V \subset X'$.
    Then define
    \begin{enumeratetheorem}
        \item $\anihu = \braces{x' \in X' \mid \forall x \in U: x'(x) = 0}$ as the annihilator of $U$ in $X'$.
        \item $\anihv = \braces{x \in X \mid \forall x' \in V: x'(x) = 0}$ as the annihilator of $V$ in $X$.
    \end{enumeratetheorem}
    In linear algebra, the annihilator is isomorphic to the orthogonal complement of a set.
    Similarly we generalize the idea of a dual structure isomorphic to the image or kernel to idk... todo
\end{definition}

\noindent To prove properties about these sets, we need another corollary
of the theorem of Hahn-Banach:

\begin{revision}
    Let $(X, \norm{}{\cdot})$ be a normed \thefield\ vector space,
    $U \subset X$ a closed subspace and $x \in X \setminus U$.
    Then we have $\exists x' \in X': x'|_U = 0 \logicaland x'(x) \neq 0$.
\end{revision}
\begin{proof}
    TODO
\end{proof}

\begin{theorem}
    Let $(X, \norm{}{\cdot})$ be a normed \thefield\ vector space, $U \subset X$ and $V \subset X'$.
    Then we have
    \begin{enumeratetheorem}
        \item $\anihu \subset X'$ and $\anihv \subset X$ are both closed linear subspaces of their respective supsets.
        \item Let $U$ be closed. Then we have $(X / U)' \isomorphto \anihu$ (they are isomorphic).
        \item Let $U$ be closed. Then we have $U' \isomorphto X' / \anihu$ (they are isomorphic).
    \end{enumeratetheorem}
\end{theorem}
\begin{proof}
    \begin{enumerateproof}
        \item 
            We will focus on $\anihu \subset X'$. The other statement works analogously. \\
            We first prove $\anihu$ is not empty:
            TODO.

            \noindent Let $x'_1, x'_2 \in X'$ and $\lambda \in \thefield$.
            We first prove $\anihu \subset X'$ is a linear subspace:
            \begin{align*}
                (x'_1 + \lambda x'_2)(x)
                &= x'_1(x) + \lambda x'_2(x)    \explainstep{substitute $x$ in} \\
                &= 0 + \lambda 0 = 0            \explainstep{use $x'_1, x'_2 \in X'$}
            \end{align*}

            \noindent Let $\sequence{k}{x'_k} \subset \anihu$ be a converging sequence. \\
            We now prove that $x'$ converges in $\anihu$:
            \begin{align*}
                todo
            \end{align*}

        \item 

        \item 

    \end{enumerateproof}
\end{proof}

\noindent In summary, TODO.

\newcommand{\kerdualperp}{\ensuremath{(\ker T')_\perp}}
\begin{theorem}
    Let $T \in \contoperators{X}{Y}$ be a bounded, linear operator.
    Then we have $\overline{\im T} = \kerdualperp$.
\end{theorem}
\begin{proof}
    \leftrightinclusion
    Let $Tx \in \im T$ with $x \in X$ and $y' \in \ker T'$. \\
    We first prove $Tx \in \kerdualperp$:
    \begin{align*}
        y'(Tx) &= (y' \chain T)(x)  \explainstep{associativity and use \chain\ notation} \\
        &= (T' y')(x)               \explainstep{apply adjoint op. definition} \\
        &= 0(x) = 0                 \explainstep{apply $y' \in \ker T'$}
    \end{align*}
    Since this holds for all choices of $Tx$ we get $\im T \subset \kerdualperp$. \\
    Since \kerdualperp is closed, we also get $\overline{\im T} \subset \kerdualperp$.

    \noindent \rightleftinclusion
    TODO
\end{proof}

% \begin{theorem}
%     TODO: The stuff with the Id - T operator.
% \end{theorem}
% \begin{proof}
%     TODO
% \end{proof}
% TODO: Idk how useful this will be!

\begin{corollary}
    Let $T \in \contoperators{X}{Y}$ be a linear, continuous operator with $\im T$ closed.
    Then we have $Tx = y$ has a solution if and only if $T'y' = 0 \implies y'(y) = 0$.
\end{corollary}
\begin{proof}
\end{proof}

\begin{example}
\end{example}

% TODO: Theorem about solutions to Tx=y
% TODO: Example as to how it relates to Lagrange Multiplicators in Optimisation
