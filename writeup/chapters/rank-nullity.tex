\subsection{The rank-nullity theorem for operators}
\newcommand{\anihv}{\ensuremath{V_\perp}}

\begin{definition}
    Let $(X, \norm{}{\cdot})$ be a normed \thefield\ vector space, $U \subset X$ and $V \subset X'$.
    Then we define the annihilator of $V$ in $X$ as
    \[ \anihv = \braces{x \in X \mid \forall x' \in V: x'(x) = 0} \]
    \note{
        In linear algebra, the annihilator is isomorphic to the orthogonal complement of a set.
        Similarly we generalize the idea of a dual structure isomorphic to the image or kernel to idk... todo
    }
\end{definition}

\noindent To prove properties about this set, we need another corollary of the theorem of Hahn-Banach:

\begin{revision}\label{revision:hahn-banach-closed-subspace}
    Let $(X, \norm{}{\cdot})$ be a normed \thefield\ vector space,
    $U \subset X$ a closed subspace and $x \in X \setminus U$.
    Then we have
    \[ \exists x' \in X': x'|_U = 0 \logicaland x'(x) \neq 0 \]
\end{revision}
\begin{proof}
    Let $Y = X / U$ be the (canonical) quotient space.
    Then $Y$ is a normed \thefield\ vector space. (todo: why?)
    Set $y = x \in Y$.
    We can apply the theorem of Hahn-Banach\ \ref{theorem:hahn-banach-op-norm}
    to obtain $y' \in Y'$ with $y'(y) \neq 0$ and $y'|_U = 0$.
\end{proof}

\begin{theorem}
    Let $(X, \norm{}{\cdot})$ be a normed \thefield\ vector space, $U \subset X$ and $V \subset X'$.
    Then we have
    \[ \anihv \subset X \text{ is a closed linear subspace} \]
\end{theorem}
\begin{proof}
    We will focus on $\anihv \subset X'$. The other statement works analogously. \\
    We first prove $\anihv$ is not empty:
    TODO.

    \noindent Let $x'_1, x'_2 \in X'$ and $\lambda \in \thefield$.
    We first prove $\anihv \subset X'$ is a linear subspace:
    \begin{align*}
        (x'_1 + \lambda x'_2)(x)
        &= x'_1(x) + \lambda x'_2(x)    \explainstep{substitute $x$ in} \\
        &= 0 + \lambda 0 = 0            \explainstep{use $x'_1, x'_2 \in X'$}
    \end{align*}

    \noindent Let $\sequence{k}{x'_k} \subset \anihv$ be a converging sequence. \\
    We now prove that $x'$ converges in $\anihv$:
    \begin{align*}
        todo
    \end{align*}
\end{proof}

\noindent In summary, TODO.

\newcommand{\kerdualperp}{\ensuremath{(\ker T')_\perp}\xspace}

\begin{theorem}\label{theorem:rank-nullity-generalized}
    Let $T \in \contoperators{X}{Y}$ be a bounded, linear operator.
    Then we have \[ \overline{\im T} = \kerdualperp \]
    \note{
        In linear algebra this is usually presented for finite-dimensional vector spaces
        (see Satz 6.1.5\ \cite{Wer18})
    }
\end{theorem}
\begin{proof}
    \leftrightinclusion
    Let $Tx \in \im T$ with $x \in X$ and $y' \in \ker T'$. \\
    We first prove $Tx \in \kerdualperp$:
    \begin{align*}
        y'(Tx) &= (y' \chain T)(x)  \explainstep{associativity and use \chain\ notation} \\
        &= (T' y')(x)               \explainstep{apply adjoint op. definition} \\
        &= 0(x) = 0                 \explainstep{apply $y' \in \ker T'$}
    \end{align*}
    Since this holds for all choices of $Tx$ we get \[ \im T \subset \kerdualperp \]
    Since \kerdualperp is closed, we also get \[ \overline{\im T} \subset \kerdualperp \]

    \noindent \rightleftinclusion
    We can prove the contraposition
    \[ \parens{Y \setminus \overline{\im T}} \subset \parens{Y \setminus \kerdualperp} \]
    Set $U = \overline{\im T}$ and let $y \in Y \setminus U$.
    We know $U$ that a closed linear subspace.
    The corollary of the theorem of Hahn-Banach\ \ref{revision:hahn-banach-closed-subspace} tells us that
    \[ \exists y' \in Y' : y'|_U = 0 \logicaland y'(y) \neq 0 \]
    Since $\ker T' \subset Y'$ 
    and \[\forall y' \in \ker T' : y'(y) \neq 0 \]
    we get \[ y \in Y \setminus \kerdualperp \]
\end{proof}

\begin{corollary}\label{corollary:operator-solutions}
    Let $T \in \contoperators{X}{Y}$ be a linear, continuous operator with $\im T$ closed. Then we have 
    \[ y \in \im T \text{ if and only if } T'y' = 0 \implies y'(y) = 0 \]
\end{corollary}
\begin{proof}
    We have
    \begin{align*}
        y \in \im T 
        &= \overline{\im T}                                     \explainstep{$\im T$ is closed} \\
        &= \kerdualperp                                         \explainstep{apply theorem \ref{theorem:rank-nullity-generalized}} \\
        \iff &\forall y' \in \ker T' : y'(y) = 0                \explainstep{apply def. of annihilator} \\
        \iff &\forall y' \in Y': T'y' = 0 \implies y'(y) = 0    \explainstep{write in equivalent way}
    \end{align*}
\end{proof}

\begin{example}
    In nonlinear optimisation theory, the lemma of Farkas states\ \cite{harks_lecture_2026}: \\
    Let $A \in \reals^{k \times n}$ and $d \in \reals^k$.
    Then exactly one of the following statements is true:
    \begin{itemize}
        \item $\exists x \ge 0 : Ax = d$
        \item $\exists \lambda \in \reals^k : \lambda A \ge 0 \logicaland \lambda d < 0$
    \end{itemize}
    The lemma is equivalent to the ability to separate two convex sets using a hyperplane.
    It is used to prove the KKT-theorem which allows for nonlinear optimisation with constraints.

    We can write down an equivalent version of Farkas lemma: \\
    Let $A \in \reals^{k \times n}$ and $d \in \reals^k$.
    Then the following statements are equivalent:
    \begin{itemize}
        \item $\exists x \ge 0 : Ax = d$
        \item $\forall \lambda \in \reals^k : A^T \lambda^T < 0 \logicalor \lambda d \ge 0$
    \end{itemize}

    \noindent Let $T = A$ and $T' = A^T$.
    This makes the lemma match the statement of corollary\ \ref{corollary:operator-solutions}.
\end{example}
