\subsection{The rank-nullity theorem for operators}
\newcommand{\anihv}{\ensuremath{V_\perp}}

\begin{definition}
    Let $(X, \norm{}{\cdot})$ be a normed \thefield\ vector space, $U \subset X$ and $V \subset X'$.
    Then we define the annihilator of $V$ in $X$ as
    \[ \anihv = \braces{x \in X \mid \forall x' \in V: x'(x) = 0} \]
    % \note{
    %     In case of $X = \reals^2$ and $V$ 1-dimensional subspace (common in linear algebra), 
    %     the annihilator is the set of straight lines through $0$ in $\reals^2$.
    % }
\end{definition}

\noindent To prove properties about this set, we need another corollary of the theorem of Hahn-Banach:

\begin{revision}\label{revision:hahn-banach-closed-subspace}
    Let $(X, \norm{}{\cdot})$ be a normed \thefield\ vector space,
    $U \subset X$ a closed subspace and $x \in X \setminus U$.
    Then we have
    \[ \exists x' \in X': x'|_U = 0 \logicaland x'(x) \neq 0 \]
\end{revision}
\begin{proof}
    Let $Y = X / U$ be the (canonical) quotient space.
    Then $Y$ is a normed \thefield\ vector space. (todo: why?)
    Set $y = x \in Y$.
    We can apply the theorem of Hahn-Banach\ \ref{theorem:hahn-banach-op-norm}
    to obtain $y' \in Y'$ with $y'(y) \neq 0$ and $y'|_U = 0$.
\end{proof}

\begin{theorem}
    Let $(X, \norm{}{\cdot})$ be a normed \thefield\ vector space, $U \subset X$ and $V \subset X'$.
    Then we have
    \[ \anihv \subset X \text{ is a closed linear subspace} \]
\end{theorem}
\begin{proof}
    We have
    \[ \anihv = \bigcap_{x' \in V}{\inv{(x')}(0)} \]
    As an intersection of closed sets, \anihv must be closed.
\end{proof}

\newcommand{\kerdualperp}{\ensuremath{(\ker T')_\perp}\xspace}

\begin{theorem}\label{theorem:rank-nullity-generalized}
    Let $T \in \contoperators{X}{Y}$ be a bounded, linear operator.
    Then we have \[ \overline{\im T} = \kerdualperp \]
    \note{
        In linear algebra lectures this is proven for finite-dimensional vector spaces
        (see Satz 6.1.5\ \cite{Wer18})
    }
\end{theorem}
\begin{proof}
    \leftrightinclusion
    Let $Tx \in \im T$ with $x \in X$ and $y' \in \ker T'$. \\
    We first prove $Tx \in \kerdualperp$:
    \begin{align*}
        y'(Tx) &= (y' \chain T)(x)  \explainstep{associativity and use \chain\ notation} \\
        &= (T' y')(x)               \explainstep{apply adjoint op. definition} \\
        &= 0(x) = 0                 \explainstep{apply $y' \in \ker T'$}
    \end{align*}
    Since this holds for all choices of $Tx$ we get \[ \im T \subset \kerdualperp \]
    Since \kerdualperp is closed, we also get \[ \overline{\im T} \subset \kerdualperp \]

    \noindent \rightleftinclusion
    We can prove the contraposition
    \[ \parens{Y \setminus \overline{\im T}} \subset \parens{Y \setminus \kerdualperp} \]
    Set $U = \overline{\im T}$ and let $y \in Y \setminus U$.
    We know $U$ that a closed linear subspace.
    The corollary of the theorem of Hahn-Banach\ \ref{revision:hahn-banach-closed-subspace} tells us that
    \[ \exists y' \in Y' : y'|_U = 0 \logicaland y'(y) \neq 0 \]
    Since $\ker T' \subset Y'$ 
    and \[\forall y' \in \ker T' : y'(y) \neq 0 \]
    we get \[ y \in Y \setminus \kerdualperp \]
\end{proof}

\begin{corollary}\label{corollary:operator-solutions}
    Let $T \in \contoperators{X}{Y}$ be a linear, continuous operator with $\im T$ closed. Then we have 
    \[ y \in \im T \text{ if and only if } \forall y' \in Y': T'y' = 0 \implies y'(y) = 0 \]
\end{corollary}
\begin{proof}
    We have
    \begin{align*}
        y \in \im T 
        &= \overline{\im T}                                     \explainstep{$\im T$ is closed} \\
        &= \kerdualperp                                         \explainstep{apply theorem \ref{theorem:rank-nullity-generalized}} \\
        \iff &\forall y' \in \ker T' : y'(y) = 0                \explainstep{apply def. of annihilator} \\
        \iff &\forall y' \in Y': T'y' = 0 \implies y'(y) = 0    \explainstep{write in equivalent way}
    \end{align*}
\end{proof}

\begin{example}
    We can refer back to the example\ \ref{example:shift-operator} with the left shift operator. \\
    The left shift operator $T$ has $\im T = \lp$, which is (trivially) closed. \\
    Since we already know the adjoint operator
    we can verify that the last theorem\ \ref{theorem:rank-nullity-generalized} works. \\
    Everything is up to isometry:
    Let $\sequence{k}{x_k} \in \lp$ such that $T\parens{\sequence{k}{x_k}} = (x_2, \dots) \in \lp$.
    Then \[ T'y' = 0 \implies y'(y) = 0 \]
    amounts to "if the right shifted version of a sequence is 0 then 
\end{example}
